% chapter3_13.tex -- de (German)
% FAQ
\newpage
\section{Software-FAQ -- oder Hinweise zu "`beliebten"' Fehlern}

In diesem Abschnitt werden nochmals stichpunktartig h�ufig gestellte
Fragen samt ihrer L�sungsm�glichkeiten aufgez�hlt. Hinweise zur
Erweiterung dieses Kapitels werden gerne per e-mail unter
\url{mailto:himself@schlizbaeda.de} entgegengenommen, am liebsten gleich
mit den dazugeh�rigen Antworten \smiley{smile}

\subsection*{Speicherkapazit�t der SD-Karte}
Eigentlich reichen 4GB zur Installation des Betriebssystems \os{Raspbian
Lite}, aber den \software{mpd} darf man dann nicht mehr kompilieren
m�ssen{\dots} (siehe Kapitel \ref{sect:mpd_recompile})\\
\uline{8\,--\,16GB:} Diese Kapazit�t ist ausreichend f�r das
Betriebssystem. Viele Erweiterungen k�nnen \textit{problemlos}
nachinstalliert werden. Zudem ist noch gen�gend Speicher frei f�r die
H�rspiele der Kinder.
\begin{bclogo}[logo = \bclampe, noborder = true]{Hinweis}
Wichtiger als die Speicherkapazit�t ist aber die Zugriffsgeschwindigkeit
auf die Karte. Die sollte mindestens Class 10 betragen. Aber auch hier
kommt es bei der gleichen Angabe zu gro�en messbaren
Geschwindigkeitsunterschieden.\\
Der {\autor} empfiehlt hier tats�chlich, zu den g�ngigen Markenprodukten 
zu greifen.
\end{bclogo}

\subsection*{\os{Raspbian Lite} oder \os{Raspbian Desktop}?}
blabla

\subsection*{Quelldateien von {\autor}s \Bezeichnung}
../files

\subsection*{framps Raspbiback pr�fen}
todo! \todo{testen!}

\subsection*{Startupsound}
Startupsound XOR mpd-Playlist!

