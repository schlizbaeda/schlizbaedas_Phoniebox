% chapter3_02.tex -- de (German)
% OnOffSHIM
\section{\onoffshim}
\label{sect:onoffshim_software}
Der {\onoffshim} ist ein kleines Erweiterungsmodul, das ein komfortables
Ein- und Ausschalten inklusive sauberem Herunterfahren des {\RPi} 
erm�glicht:\\
\url{https://shop.pimoroni.com/search?type=product&q=onoffshim}\\
Prinzipiell ist dieses Modul vom Konzept her gut durchdacht
und f�r batteriebetriebene \RPi-Projekte gut geeignet. Allerdings f�hrte
bei mir ein zu langes Bet�tigen des On-/Off-Tasters dazu, dass der
{\onoffshim} die Spannungsversorgung nicht vollst�ndig ausschaltete,
siehe Abschnitt \ref{sect:onoffshim_relay}.

\begin{bclogo}[arrondi = 0.2, logo = \bcinfo, ombre = true, epOmbre = 0.25, couleurOmbre = black!30,blur]{Achtung}
Die hier beschriebene Installation der Software verwendet \textbf{nicht}
die Originalsoftware des Herstellers \textit{Pimoroni}, sondern eine
Variante, die unter
\begin{smaller}\url{https://retropie.org.uk/forum/topic/15727/tutorial-onoff-shim-exposed-neat-powerswitch-from-pimoroni/}\end{smaller}\\
beschrieben wird!
\end{bclogo}

Vom Hersteller \textit{Pimoroni} wird zwar auch eine Software
mitgeliefert, die als GNU/Linux-Daemon (Hintergrunddienst) arbeitet und
mit folgendem Kommando aus dem Internet heruntergeladen und installiert
werden k�nnte:\\
\cmdPi{curl https://get.pimoroni.com/onoffshim | bash}
\comment{wurde so nicht durchgef�hrt!}

Diese Software war mir aber zu undurchsichtig, so dass ich lieber das
einfachere und (f�r mich) durchschaubare Verfahren des Anwenders 
\textit{cyberghost} aus dem Retropie-Forum 
(\url{https://retropie.org.uk/}) verwendete. Mir ist an dieser Stelle
wichtig zu betonen, dass auch die Software von \textit{Pimoroni}
einwandfrei funktioniert und ebenso verwendet werden k�nnte.

Grunds�tzlich ist der Einschaltvorgang beim {\onoffshim} ein rein
hardwarem��ig umgesetztes Konzept, das keinerlei zus�tzliche Software
ben�tigt. Beim Ausschalten ist eine Software erforderlich, die die
Bet�tigung des Tasters erkennt und das Betriebssystem vor dem Abschalten
der Versorgungsspannung sauber beendet. Die Variante aus dem
Retro\-pie-Forum 
%\begin{smaller}\url{https://retropie.org.uk/forum/topic/15727/tutorial-onoff-shim-exposed-neat-powerswitch-from-pimoroni/}\end{smaller}
besteht aus lediglich zwei Shellskripten:

\begin{table}[!h]
\centering
\renewcommand{\arraystretch}{1.25}
\begin{tabular}{|p{0.35\textwidth}|p{0.2\textwidth}|p{0.36\textwidth}|}
\hline
\textbf{Skript}							&	\textbf{Aufruf durch}	&	\textbf{Speicherort}\\
\hline
\filenam{onoffshim\_switch.sh}			&	\filenam{/etc/rc.local}	&	\filenam{/home/pi}\\
\hline
\filenam{onoffshim\_gpio-shutoff.sh}	&	\software{systemd}		&	\filenam{/lib/systemd/system-shutdown}\\
\hline
\end{tabular}
\vspace{0.5cm}
\caption{Skripte f�r den \onoffshim}
\label{tab:onoffshim_scripts}
\end{table}


\subsection{Installation der Shellskripte f�r den \onoffshim}
Die beiden Shellskripte aus Tabelle \ref{tab:onoffshim_scripts} k�nnen
vom PC aus dem Downloadverzeichnis dieser Dokumentation mit dem
Kommando\\
\cmdPC{scp ./files/onoffshim/*.sh pi@phoniebox1:/home/pi} in das
Homeverzeichnis des {\RPi} kopiert werden.

Das Skript \filenam{onoffshim\_switch.sh} muss auf dem {\RPi} beim
Hochfahren gestartet werden. Dazu kann es im Homeverzeichnis des
Benutzers verbleiben, muss aber mit Ausf�hrrechten versehen werden:\\
\cmdPi{chmod 755 onoffshim\_switch.sh}\\

Das Skript \filenam{onoffshim\_gpio-shutoff.sh} dagegen soll beim
Herunterfahren des {\RPi} �ber \software{systemd} aufgerufen werden.
Dazu muss es mit Rootrechten in das Verzeichnis
\filenam{/lib/systemd/system-shutdown} kopiert werden, wo es von
\software{systemd} gesucht und ausgef�hrt wird:\\
\cmdPi{sudo mv onoffshim\_gpio-shutoff.sh /lib/systemd/system-shutdown}\\
\cmdPi{sudo chown root:root /lib/systemd/system-shutdown/onoffshim\_gpio-shutoff.sh}\\
\cmdPi{sudo chmod 755 /lib/systemd/system-shutdown/onoffshim\_gpio-shutoff.sh}

Nicht zwingend erforderlich, aber zur Bewahrung der �bersicht lohnt es
sich, im Homeverzeichnis einen symbolischen Link zum neuen Speicherort
des Skriptes \filenam{onoffshim\_gpio-shutoff.sh} anzulegen:\\
\cmdPi{ln -s /lib/systemd/system-shutdown/onoffshim\_gpio-shutoff.sh}


\subsection{Eintr�ge in den aufrufenden Dateien vornehmen}
Damit das Skript \filenam{onoffshim\_switch.sh} beim Hochfahren des
{\RPi} gestartet wird, muss es in die Datei \filenam{/etc/rc.local}
eingetragen werden:\\
\cmdPi{sudo nano /etc/rc.local}\\
\editor{\#!/bin/sh -e\\
\#\\
\# rc.local\\
\#\\
\# This script is executed at the end of each multiuser runlevel.\\
\# Make sure that the script will \dq exit 0\dq\ on success or any other\\
\# value on error.\\
\#\\
\# In order to enable or disable this script just change the execution\\
\# bits.\\
\#\\
\# By default this script does nothing.\\
\\
\# Print the IP address\\
\vdots\\
%\_IP=\$(hostname -I) || true\\
%if \[ "\$\_IP" \]\; then\\
%  printf "My IP address is \%s\bs n" "\$\_IP"\\
%fi\\
\\
\textcolor{red}{\# added by schlizb�da:\\
/home/pi/onoffshim\_switch.sh \& \# do it in the background!}\\
\\
exit 0}


\subsection{Funktionsweise}
Das Skript \filenam{onoffshim\_switch.sh} wird beim Hochfahren �ber den
Eintrag in \filenam{/etc/rc.local} im Hintergrund gestartet und l�uft
solange in einer Schleife, bis der Taster des {\onoffshim} an GPIO17
(Pin 11 der Stiftleiste) bet�tigt wird. Damit wird die Schleife
verlassen und das Herunterfahren des {\RPi} durch Aufruf des Kommandos
\texttt{poweroff} veranlasst.\\
\software{systemd} f�hrt den {\RPi} herunter. Im laufe dieses Prozesses
werden alle ausf�hrbaren Dateien im Verzeichnis
\filenam{/lib/systemd/system-shutdown} gestartet, in dem sich unser
Shellskript \filenam{onoffshim\_gpio-shutoff.sh} befindet:\\
Zun�chst wird �ber den GPIO27 (Pin 13, Variable \$cut\_pin) das Relais
angesteuert, um den Taster des {\onoffshim} k�nstlich zu trennen (siehe
Kapitel \ref{sect:onoffshim_relay}). Der GPIO17 (Pin 11, Variable
\$led\_pin) wird auf Ausgang umkonfiguriert, um zun�chst in einer
for-Schlei\-fe die rote LED auf dem {\onoffshim} dreimal blinken zu
lassen. Schlie�lich wird der GPIO4 (Pin 7, Variable \$poweroff\_pin) als
Ausgang konfiguriert und auf low geschaltet, damit der {\onoffshim} die
Versorgungsspannung abschaltet.

Siehe auch:\\
\begin{scriptsize}\url{https://forum-raspberrypi.de/forum/thread/45820-phoniebox-2-0-rc7-mpd-spielt-ueber-rfid-karte-gewaehlte-musik-nicht-ab/?postID=413400#post413400}\end{scriptsize}
