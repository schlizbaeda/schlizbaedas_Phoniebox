% chapter3_10.tex -- de (German)
% auto start
\newpage
\section{\software{systemd:} Autostart der \Bezeichnung-Software einrichten}
\label{sect:systemd}
Quelle:\\
\url{https://github.com/MiczFlor/RPi-Jukebox-RFID/wiki/CONFIGURE-stretch#auto-start-the-phoniebox}

Im folgenden Abschnitt wird die {\Bezeichnung} so eingerichtet, dass die
gesamte Software beim Hochfahren gestartet wird. Dies geschieht �ber
den Systemdienst \software{systemd}

\begin{bclogo}[logo = \bclampe, noborder = true]{Hinweis}
Die in den folgenden Kommandos angegebenen Dateien wurden zwar f�r
\os{Raspbian Stretch} vorgesehen, aber auch dort kam schon
\software{systemd} zum Einsatz. Daher ist es unproblematisch, dass hier
die \os{Stretch}-Dateien f�r \os{Raspbian Buster} verwendet werden.
\textcolor{red}{Wichtig ist aber, dass in Abschnitt \ref{sect:apt-get}
auch die entsprechenden Pakete f�r \software{Python3} installiert
wurden!} 
\end{bclogo}


\cmdPi{\begin{scriptsize}sudo cp /home/pi/RPi-Jukebox-RFID/misc/sampleconfigs/phoniebox-rfid-reader.service.stretch-default.sample\\ /etc/systemd/system/phoniebox-rfid-reader.service\end{scriptsize}}\\
\cmdPi{\begin{scriptsize}sudo cp /home/pi/RPi-Jukebox-RFID/misc/sampleconfigs/phoniebox-startup-sound.service.stretch-default.sample\\ /etc/systemd/system/phoniebox-startup-sound.service\end{scriptsize}}\\
\cmdPi{\begin{scriptsize}sudo cp /home/pi/RPi-Jukebox-RFID/misc/sampleconfigs/phoniebox-gpio-buttons.service.stretch-default.sample\\ /etc/systemd/system/phoniebox-gpio-buttons.service\end{scriptsize}}\\
\cmdPi{\begin{scriptsize}sudo cp /home/pi/RPi-Jukebox-RFID/misc/sampleconfigs/phoniebox-idle-watchdog.service.sample\\ /etc/systemd/system/phoniebox-idle-watchdog.service\end{scriptsize}}

Neue Dienste bekannt geben:\\
\cmdPi{sudo systemctl daemon-reload}

Dienste aktivieren:\\
\cmdPi{sudo systemctl enable phoniebox-idle-watchdog}\\
\cmdPi{sudo systemctl enable phoniebox-rfid-reader}\\
\cmdPi{sudo systemctl enable phoniebox-startup-sound}\\
\cmdPi{sudo systemctl enable phoniebox-gpio-buttons}\\
\stdout{Created symlink /etc/systemd/system/multi-user.target.wants/\textcolor{red}{<Name>}.service --> /etc/systemd/system/\textcolor{red}{<Name>}.service.}

Dienste starten (anstelle eines Reboots):\\
\cmdPi{sudo systemctl start phoniebox-idle-watchdog}\\
\cmdPi{sudo systemctl start phoniebox-rfid-reader}\\
\cmdPi{sudo systemctl start phoniebox-startup-sound}\\
\cmdPi{sudo systemctl start phoniebox-gpio-buttons}

Status der Dienste anzeigen:\\
\cmdPi{sudo systemctl status phoniebox-idle-watchdog}\\
\cmdPi{sudo systemctl status phoniebox-rfid-reader}\\
\cmdPi{sudo systemctl status phoniebox-startup-sound}\\
\cmdPi{sudo systemctl status phoniebox-gpio-buttons}

\begin{bclogo}[logo = \bclampe, noborder = true]{Hinweis}
\textcolor{red}{Sollten hier Fehler auftreten, so liegt die Ursache
meist darin, dass nicht von allen Modulen die jeweils aktuelle Version
heruntergeladen und installiert wurde. Gerade die Umstellung auf
Python3 ist noch nicht in allen Teilbeschreibungen be�cksichtigt und
es wurden nur die Python2-Module installiert} 
\end{bclogo}

\cmdPi{sudo reboot}

Jetzt sollten alle erforderlichen Softwarepakete auch im
\textit{Autostart} richtig hinterlegt sein. Beim Hochfahren ert�nt ein
Startupsound. Danach sollte sowohl der {\reader} auf aufgelegte
{\Karte}n reagieren, als auch die Funktion aller GPIO-Taster gegeben
sein.\\
Die Original-Software der {\Bezeichnung} sollte weitestgehend richtig
und sauber installiert sein. Es fehlt nur noch die aus einem anderen
Repository stammende externe Software zur Ansteuerung des OLED-Displays.
