% chapter3_06.tex -- de (German)
% installation of the RFID reader
\newpage
%\begin{bclogo}[logo = \bclampe, noborder = true]{Hinweis}
\begin{bclogo}[arrondi = 0.2, logo = \bcinfo, ombre = true, epOmbre = 0.25, couleurOmbre = black!30,blur]{Achtung}
Ab hier bezieht sich diese Dokumentation auf die Seite\\
\url{https://github.com/MiczFlor/RPi-Jukebox-RFID/wiki/CONFIGURE-stretch}\\
von MiczFlor!
\end{bclogo}

\section{{\reader} installieren}
In diesem Abschnitt wird die Installation des \textit{Neuftech USB
RFID-Reader 125kHz} beschrieben, der in {\autor}s {\Bezeichnung}
verbaut wurde und bei amazon erh�ltlich ist:\\
\url{https://www.amazon.de/Neuftech-Reader-Kartenleseger%C3%A4t-Kartenleser-Kontaktlos/dp/B018OYOR3E}

Dieses Ger�t emuliert am USB-Anschluss eine Tastatur (ein Ger�t aus der
USB-Klasse \textit{Human Interface Device},
\url{https://de.wikipedia.org/wiki/Human_Interface_Device}).\\
Die \Bezeichnung-Soft\-ware f�ngt Ereignisse vom {\reader} jedoch �ber
\software{udev} ab:\\
\url{https://wiki.ubuntuusers.de/udev/}


\subsection{{\reader} �berpr�fen}
Einen unter \os{Raspbian} korrekt installierten {\reader} erkennt man
durch Kontrolle folgender Punkte
\begin{compactitem}
\item{Die LED des {\reader}s leuchtet -- im eingebauten Zustand evtl. nicht erkennbar}
\item{Piepton beim Hochfahren}
\item{Piepton beim Lesen einer \Karte}
\item{�berpr�fen, ob das Softwareevent f�r den {\reader} ordnungsgem��
      eingerichtet ist:\\
      \cmdPi{ls -la /dev/input/by-id}\\
      \stdout{total 0\\
              drwxr-xr-x 2 root root  60 Jan 11 15:36 .\\
              drwxr-xr-x 4 root root 120 Jan 11 15:36 ..\\
              lrwxrwxrwx 1 root root   9 Jan 11 15:36 usb-HXGCoLtd\_27db-event-kbd -> ../event0}
     }
\end{compactitem}


\subsection{{\reader} in der \Bezeichnung-Software registrieren}
\cmdPi{cd /home/pi/RPi-Jukebox-RFID/scripts}\\
\cmdPi{python3 RegisterDevice.py}\comment{liefert \zB:}\\
\stdout{Choose the reader from list\\
0 HID 046a:0011\\
\textcolor{red}{1 HXGCoLtd Keyboard}\\
Device Number: \textbf{\textcolor{red}{1}}}\\
Unter GNU/Linux wird der Chipsatz des {\reader}s als \textit{HXGCoLtd
Keyboard} erkannt. Daher ist bei der Abfrage der \texttt{Device Number}
in diesem Beispiel der Wert \textcolor{red}{1} einzugeben.

Ob die Registrierung des {\reader}s erfolgreich war, kann mit der Datei\\
\filenam{/home/pi/RPi-Jukebox-RFID/scripts/deviceName.txt} gepr�ft
werden. Diese Datei enth�lt den oben gew�hlten Ger�tenamen:\\
\cmdPi{cat deviceName.txt}\\
\stdout{HXGCoLtd Keyboard}


\subsection{RFID-Konfigurationsdatei der \Bezeichnung-Software anlegen}
\label{sect:RFID_cfg}
\cmdPi{cd /home/pi/RPi-Jukebox-RFID/settings}\\
\cmdPi{cp rfid\_trigger\_play.conf.sample rfid\_trigger\_play.conf}\\
\cmdPi{sudo chown pi:pi rfid\_trigger\_play.conf}\\ % braucht's das?
\cmdPi{sudo chmod 665 rfid\_trigger\_play.conf}     % braucht's das?












