% chapter3_05.tex -- de (German)
% configuration of mpd
\newpage
\section{\software{Music Player Daemon (mpd)} einrichten}
Die gesamte Konfiguration des \software{mpd} wird in der Datei
\filenam{/etc/mpd.conf} vorgenommen. Diese Datei besteht aus Eintr�gen
in der Form \textit{Schl�sselwort Wert} und in der Standardauslieferung
aus vielen Kommentarzeilen. Letztlich m�ssen folgende Eintr�ge erg�nzt
oder angepasst werden:\\
\cmdPi{sudo nano /etc/mpd.conf}\\
\editor{music\_directory    "{}/home/pi/RPi-Jukebox-RFID/shared/audiofolders"\\
playlist\_directory "{}/home/pi/RPi-Jukebox-RFID/playlists"\\
user               "root"\\
auto\_update        "yes"\\ % (you have to remove the \# in front of that line)\\
auto\_update\_depth  "10"\\ % (remove the \# and change the value to 10)\\
%\#mixer\_control      "yourAudioIfaceNameHere"\\ % (you need to uncomment this line and change the audio iFace shortname)\\
mixer\_control      "Master"
}
        
Folgendes Kommando dient dazu, die Bezeichnung f�r den Eintrag
\texttt{mixer\_control} herauszufinden:\\
\cmdPi{amixer scontrols}\comment{liefert z.B. bei \miniamp-Installation}\\
\stdout{Simple mixer control 'Master',0}

Der \software{mpd} wird mit der neuen Konfiguration mit folgendem
Kommando gestartet:\\
\cmdPi{mpc update}

\textbf{Test von \software{mpd}}\\
Jetzt ist der Zeitpunkt gekommen, einen geschmeidigen Audiotest mit
\software{mpd} durchzuf�hren. Dazu zun�chst vom PC eine Audiodatei
auf den {\RPi} in das oben konfigurierte Audioverzeichnis
\filenam{/home/pi/RPi-Jukebox-RFID/shared/audiofolders} kopieren und
mit dem Client \software{mpc} abspielen:\\
\cmdPC{scp /Pfad/Musik.flac pi@phoniebox1:/home/pi/RPi-Jukebox-RFID/shared/audiofolders}

\cmdPi{mpc add /home/pi/RPi-Jukebox-RFID/shared/audiofolders/Musik.flac}\\
\cmdPi{mpc play}

Sollte keine Musik h�rbar sein, mit dem \software{alsamixer} die
Lautst�rke �berpr�fen und anpassen:\\
\cmdPi{alsamixer}

(siehe Abbildung \ref{fig:alsamixer})

