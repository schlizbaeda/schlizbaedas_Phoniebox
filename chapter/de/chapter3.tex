% chapter3.tex -- de (German)
\chapter{Installation der Software}
Eigentlich ist die Installation der Software \textit{babyleicht} und man
muss (Achtung: Unwort!) \textit{nur} den \cmd{one-line-installer} von
\textit{MiczFlor} auf einem jungfr�ulichen \os{Raspbian} starten\dots\\
\url{https://github.com/MiczFlor/RPi-Jukebox-RFID#installation}

\textbf{Aber:}\\
Aufgrund vieler Anfragen der mittlerweile relativ gro�en Community rund
um die {\Bezeichnung} und der daraus resultierenden zahlreichen
\textit{Push Requests} auf github und auch wegen vieler anderer guter
Gr�nde schleichen sich in die an sich gro�artige Arbeit von
\textit{MiczFlor} naturbedingt immer wieder kleinere Fehler ein, die
insbesondere f�r unbedarfte Neulinge bisweilen schwer zu finden und zu
beseitigen sind.

%\begin{bclogo}[logo = \bclampe, noborder = true]{Hinweis}
F�r die Installation der Software m�ssen einige Konfigurationsdateien
angepasst werden. Im git-Repository unter
\url{https://github.com/schlizbaeda/schlizbaedas_Phoniebox} gibt es das
Unterverzeichnis \filenam{files}, in dem diese Dateien in dem Stand
enthalten sind, wie sie auf {\autor}s {\Bezeichnung} produktiv
eingesetzt werden. %Anstatt die Konfigurationsdateien selbst zu
%editieren, wie im diesem Kapitel beschrieben, k�nnen auch diese Vorlagen
%verwendet werden.
%\end{bclogo}
\begin{table}[h]
\centering
\renewcommand{\arraystretch}{1.11}
\begin{tabular}{|p{0.55\textwidth}|p{0.35\textwidth}|p{0.1\textwidth}|}
\hline
\textbf{/absoluter Pfad/Dateiname}	&	\textbf{Verwendung}								&	\textbf{wo?}\\
\hline
\filenam{/boot/wpa-supplicant.conf}	&	t.b.d.	&	\\
\hline
\filenam{/boot/config.txt}			&	allgemeine Konfigurationsdatei\newline des \RPi	&	link\\
\hline
\filenam{/etc/asound.conf}			&	t.b.d.	&	\\
\hline
\filenam{/home/pi/RPi-Jukebox-RFID/requirements.txt}	&	t.b.d.	&	\\
\hline
\filenam{/etc/lighttpd/lighttpd.conf}	&	Web-App der \Bezeichnung	&	\\
\hline
\filenam{/etc/sudoers}					&	t.b.d.	&	\\
\hline
\filenam{/etc/lighttpd/conf-available/15-fastcgi-php.conf}	&	t.b.d.	&	\\
\hline
\filenam{/etc/mpd.conf}				&	t.b.d.	&	\\
\hline
\filenam{/home/pi/RPi-Jukebox-RFID/scripts/gpio-buttons.py}	&	t.b.d.	&	\\
\hline
\filenam{rfid\_trigger\_play.conf}	&	In dieser Datei ist die Syntax erkl�rt	&	\\
\hline
\filenam{/etc/rc.local}					&	Autostart: Starten von \filenam{onoffshim\_switch.sh}	&	\\
\hline
\filenam{/home/pi/onoffshim\_switch.sh}	&	{\onoffshim} cyberghost	Teil 1	&	\\
\hline
\filenam{.../onoffshim\_gpio-shutoff.sh}		&	{\onoffshim} cyberghost	Teil 2	&	\\
\hline
\end{tabular}
\vspace{0.5cm}
\caption{Konfigurationsdateien von {\autor}s \Bezeichnung}
\label{tab:cfgfiles}
\end{table}
\todo{Tabelle vervollst�ndigen}TODO:

% chapter3_01.tex -- de (German)
% installation of Raspbian Buster Lite
\section{\os{Raspbian Buster Lite} auf dem {\RPi} installieren}
Einige Tutorials im Internet empfehlen zwar, die Software f�r die
{\Bezeichnung} unter \os{Raspbian Buster with desktop} zu installieren,
aber da es sich bei der {\Bezeichnung} um ein einfaches
stand\-alone-System ohne Bildschirm handelt, ist es in meinen Augen
v�llig ausreichend, die Installation auf dem wesentlich kleineren
\os{Raspbian Lite} vorzunehmen. Ein grafischer X11-Desktop ist
schlichtweg nicht erforderlich. Au�erdem reicht f�r \os{Raspbian Buster
Lite} zur Not sogar eine SD-Karte mit einer Speicherkapazit�t von 4GB.

\subsection{Erstellen einer SD-Karte mit \os{Raspbian Buster Lite}}
Zun�chst wird die aktuelle Version von \os{Raspbian Buster Lite} von der
Homepage der {\foundation}
(\url{https://www.raspberrypi.org/downloads/raspbian/}, ca. 450MiB)
heruntergeladen und entpackt (1,8GiB). Das entpackte Image wird mit
einem daf�r vorgesehenen Programm wie \software{win32diskimager} oder
am GNU/Linux-PC mit dem systemeigenen Kommando \cmd{dd} auf die SD-Karte
\textit{geflasht}, um das Image 1:1 zu �bertragen.
\begin{bclogo}[arrondi = 0.2, logo = \bcinfo, ombre = true, epOmbre = 0.25, couleurOmbre = black!30,blur]{Achtung}
Es reicht nicht, die Imagedatei einfach auf eine bereits formatierte
SD-Karte zu kopieren!
\end{bclogo}
Die Details des \textit{Flashens} von Betriebssystem-Images auf eine
SD-Karte werden von der {\foundation} unter
\url{https://www.raspberrypi.org/documentation/installation/installing-images/README.md}
ausf�hrlich beschrieben und in diesem Dokument als bekannt
vorausgesetzt.

Nach dem Flashen enth�lt die SD-Karte zwei Partitionen:
\begin{compactitem}
\item{die FAT32-formatierte \filenam{/boot}-Partition}
\item{die ext4-formatierte Rootpartition von \os{Raspbian}}
\end{compactitem}
\begin{bclogo}[logo = \bclampe, noborder = true]{Hinweis}
Beim (erneuten) Anstecken der SD-Karte am PC werden diese beiden
Partitionen im Dateimanager angezeigt. Sollte auf dem PC jedoch das
Betriebssystem \os{Windows} verwendet werden, so wird nur die
FAT32-formatierte \filenam{/boot}-Partition erkannt.
\end{bclogo}


\newpage
\subsection{Anmeldung am {\RPi} �ber \software{ssh}}
\begin{bclogo}[logo = \bclampe, noborder = true]{Hinweis}
F�r die ersten Schritte muss die Netzwerkverbindung zum {\RPi} �ber ein
verkabeltes LAN aufgebaut werden, da die WLAN-Verbindung noch nicht
eingerichtet ist! Dies geschieht erst in Abschnitt \ref{sect:setupWLAN}.
\end{bclogo}
\begin{figure}[h]
\centering
\includegraphics[width=0.90\textwidth, angle=0]{software/login.png}
\caption{erster Login �ber \software{ssh}}
\label{fig:ssh_login}
\end{figure}
Da an den {\RPi} f�r die {\Bezeichnung} weder Bildschirm noch Tastatur
angeschlossen werden, muss der Zugriff auf den {\RPi} von Anfang an �ber
\software{ssh} erfolgen. Dazu muss auf der Partition \filenam{/boot} der
SD-Karte die leere Datei \filenam{ssh} angelegt werden.\\
Nun wird die SD-Karte in den {\RPi} gesteckt, der {\RPi} \textbf{�ber
Ethernet ans LAN} geh�ngt und eingeschaltet. Nach sp�testens einer Minute
sollte \os{Raspbian Buster Lite} vollst�ndig gebootet und der {\RPi} im
LAN bekannt sein. Auf dem PC wird in einem Terminalfenster
(unter \os{Windows} in der Eingabeaufforderung \cmd{cmd.exe}) die
Software \software{ssh} mit folgendem Kommando gestartet:

\cmdPC{ssh pi@raspberrypi}\comment{Das Passwort lautet \texttt{raspberry}}

Die Bildschirmausgabe sieht in etwa wie in Abbildung \ref{fig:ssh_login}
aus.


\newpage
\subsection{\os{Raspbian} konfigurieren}
Nach erfolgreichem \software{ssh}-Login muss das System jetzt auf einen
aktuellen und sicheren Stand gebracht werden:\\
\cmdPi{sudo apt update \&\& sudo apt upgrade}\comment{System auf den neuesten Stand bringen}\\
\cmdPi{sudo raspi-config}\\
\stdout{1 Change User Password \ \ \ \ \ \ \ \ \ \ \ \ \ \ \ \ \ \ \ \ \ \ \ \ \ \ \ }\comment{Passwort unbedingt �ndern!}\\
\stdout{2 Network Options \ \ \ \ \ --> N1 Hostname \ \ \ \ \ \ \ \ \ \ \ \ }\comment{\zB \texttt{phoniebox1}}\\
\stdout{3 Boot Options \ \ \ \ \ \ \ \ --> B1 Desktop / CLI \ \ \ \ \ \ \ \ --> B2 Console Autologin}\\
%\stdout{4 Localisation Options --> I1 Change Locale \ \ \ \ \ \ \ }\comment{Landeseinstellungen}\\
\stdout{4 Localisation Options --> I2 Change Timezone \ \ \ \ \ }\comment{Zeitzone anpassen}\\
\stdout{\textcolor{white}{ \ \ \ \ \ \ \ \ \ \ \ \ \ \ \ \ \ \ \ \ \ \ } --> I4 Change Wi-fi Country }\comment{diese Anpassung ist ganz wichtig!}\\
\stdout{5 Interfacing Options \ --> P2 SSH}

Beim Verlassen von \software{raspi-config} sollte die Abfrage 
\texttt{Would you like to reboot now?} mit \button{yes} beantwortet
werden. Der {\RPi} wird neu hochgefahren. Vom PC aus kann nach ca.
einer Minute Wartezeit ein neuer \software{ssh}-Login auf den neuen
Hostnamen mit dem (hoffentlich) ge�nderten Passwort erfolgen:\\
\cmdPC{ssh pi@phoniebox1}

\subsection{WLAN einrichten}
\label{sect:setupWLAN}
Da die {\Bezeichnung} als tragbares Ger�t f�r's Kinderzimmer konzipiert
ist, sollte ein WLAN-Zu\-gang eingerichtet werden, um als Eltern sp�ter
\zB per Laptop oder Smartphone jederzeit einen \textit{bequemen}
Zugriff auf die Box zu haben. Wie h�ufig unter \os{GNU/Linux} gibt es
auch f�r die WLAN-Konfiguration mehrere zielf�hrende Wege, wie sie \zB
im Elek\-tro\-nik-Kom\-pen\-dium unter
\url{https://www.elektronik-kompendium.de/sites/raspberry-pi/1912221.htm}
beschrieben werden. F�r {\autor}s {\Bezeichnung} wird die dort
beschriebene "`Variante 2"' mittels \software{wpa\_supplicant}
verwendet.

Auf einem frisch installierten \os{Raspbian Lite} wird der WLAN-Zugriff
in der \software{ssh}-Konsole  �ber die Datei
\filenam{ /etc/wpa\_supplicant/wpa\_supplicant.conf} eingerichtet. Die
drei ersten Zeilen sollten in der Datei bereits vorhanden sein. F�r
jedes gew�nschte WLAN muss ein Eintrag \texttt{network\{\dots\}} nach
folgendem Schema eingef�gt werden. Dazu wird der auch �ber
\software{ssh} funktionierende Texteditor \software{nano} verwendet:\\
\cmdPi{sudo nano /etc/wpa\_supplicant/wpa\_supplicant.conf}\\
\editor{ctrl\_interface=DIR=/var/run/wpa\_supplicant GROUP=netdev\\
        update\_config=1\\
        country=DE\\
        \\
        network=\{ \# Erweiterungen von \autor\\
        \textcolor{white}{\ \ \ \ }ssid=\dq{<wlan-name>}\dq\ \# Klartext-Bezeichnung des Netzwerkes\\
        \textcolor{white}{\ \ \ \ }psk=\dq{<passphrase>}\dq\ \# Passwort f�r WLAN-Zugriff\\
        \textcolor{white}{\ \ \ \ }key\_mgmt=WPA-PSK\\
        \}
       }
       
Um diese Einstellungen tats�chlich zu aktivieren, m�ssen im Anschluss
folgende Kommandos abgesetzt werden:\\
\cmdPi{sudo systemctl restart dhcpcd}\\
\stdout{\textcolor{red}{Warning:} The unit file, source configuration
        file or drop-ins of dhcpcd.service changed on disk. Run
        'systemctl daemon-reload' to reload units.}\\
\cmdPi{sudo systemctl daemon-reload}\comment{gem�� obiger Warnung}\\
\cmdPi{ip addr}\\
\stdout{1: lo: <LOOPBACK\,UP\,LOWER\_UP> mtu 65536 qdisc noqueue state UNKNOWN group default qlen 1000\\
        \textcolor{white}{\ \ \ \ }link/loopback 00:00:00:00:00:00 brd 00:00:00:00:00:00\\
        \textcolor{white}{\ \ \ \ }inet 127.0.0.1/8 scope host lo\\
        \textcolor{white}{\ \ \ \ \ \ \ }valid\_lft forever preferred\_lft forever\\
        \textcolor{white}{\ \ \ \ }inet6 ::1/128 scope host\\
        \textcolor{white}{\ \ \ \ \ \ \ }valid\_lft forever preferred\_lft forever\\
        2: eth0: BROADCAST\,MULTICAST\,UP\,LOWER\_UP> mtu 1500 qdisc pfifo\_fast state UP group default qlen 1000\\
        \textcolor{white}{\ \ \ \ }link/ether b8:27:eb:36:6f:a7 brd ff:ff:ff:ff:ff:ff\\
        \textcolor{white}{\ \ \ \ }inet 192.168.178.153/24 brd 192.168.178.255 scope global dynamic noprefixroute eth0\\
        \textcolor{white}{\ \ \ \ \ \ \ }valid\_lft 859570sec preferred\_lft 751570sec\\
        \textcolor{white}{\ \ \ \ }inet6 fe80::3a9f:9d8c:1f44:e940/64 scope link \\
        \textcolor{white}{\ \ \ \ \ \ \ }valid\_lft forever preferred\_lft forever\\
        3: wlan0: <BROADCAST\,MULTICAST\,UP\,LOWER\_UP> mtu 1500 qdisc pfifo\_fast state UP group default qlen 1000\\
        \textcolor{white}{\ \ \ \ }link/ether b8:27:eb:63:3a:f2 brd ff:ff:ff:ff:ff:ff\\
        \textcolor{white}{\ \ \ \ }inet 192.168.178.154/24 brd 192.168.178.255 scope global dynamic noprefixroute wlan0\\
        \textcolor{white}{\ \ \ \ \ \ \ }valid\_lft 863877sec preferred\_lft 755877sec\\
        \textcolor{white}{\ \ \ \ }inet6 fe80::b04c:315b:a4c9:3c9d/64 scope link\\
        \textcolor{white}{\ \ \ \ \ \ \ }valid\_lft forever preferred\_lft forever
       }

Damit ist die {\Bezeichnung} f�r den k�nftigen WLAN-Zu\-griff
freigeschaltet. Der bis jetzt verwendete LAN-Zu\-griff �ber das
Ethernetkabel ist ab sofort nicht mehr notwendig. Zum Test wird die
laufende \software{ssh}-Sitzung beendet, das Ethernetkabel abgesteckt
und eine neue \software{ssh}-Sitzung gestartet:\\
\cmdPi{exit}\comment{Schlie�en der bestehenden ssh-Sitzung �ber LAN}\\
\textit{Jetzt das Ethernetkabel abstecken!}\\
\cmdPC{ssh pi@phoniebox1}\comment{neue ssh-Sizung �ber WLAN �ffnen}

%\begin{bclogo}[arrondi = 0.2, logo = \bcinfo, ombre = true, epOmbre = 0.25, couleurOmbre = black!30,blur]{Achtung}
\begin{bclogo}[logo = \bclampe, noborder = true]{Hinweis}
Entgegen vieler Tutorials wurde bei {\autor}s {\Bezeichnung} auf die
Zuteilung einer statischen (festen) IP-Adresse verzichtet! Stattdessen
wird von der {\Bezeichnung} beim Bootvorgang �ber DHCP eine dynamische
IP-Adresse angefordert, um den Netzwerkzugriff m�glichst flexibel zu
halten. Die konkret vergebene IP-Adresse kann mit dem Kommando
\cmd{ip addr} ermittelt werden und lautet im obigen Beispiel
\texttt{192.168.178.154}.
\end{bclogo}

% chapter3_02.tex -- de (German)
% OnOffSHIM
\section{\onoffshim}
\label{sect:onoffshim_software}
Der {\onoffshim} ist ein kleines Erweiterungsmodul, das ein komfortables
Ein- und Ausschalten inklusive sauberem Herunterfahren des {\RPi} 
erm�glicht:\\
\url{https://shop.pimoroni.com/search?type=product&q=onoffshim}\\
Prinzipiell ist dieses Modul vom Konzept her gut durchdacht
und f�r batteriebetriebene \RPi-Projekte gut geeignet. Allerdings f�hrte
bei mir ein zu langes Bet�tigen des On-/Off-Tasters dazu, dass der
{\onoffshim} die Spannungsversorgung nicht vollst�ndig ausschaltete,
siehe Abschnitt \ref{sect:onoffshim_relay}.

\begin{bclogo}[arrondi = 0.2, logo = \bcinfo, ombre = true, epOmbre = 0.25, couleurOmbre = black!30,blur]{Achtung}
Die hier beschriebene Installation der Software verwendet \textbf{nicht}
die Originalsoftware des Herstellers \textit{Pimoroni}, sondern eine
Variante, die unter
\begin{smaller}\url{https://retropie.org.uk/forum/topic/15727/tutorial-onoff-shim-exposed-neat-powerswitch-from-pimoroni/}\end{smaller}\\
beschrieben wird!
\end{bclogo}

Vom Hersteller \textit{Pimoroni} wird zwar auch eine Software
mitgeliefert, die als GNU/Linux-Daemon (Hintergrunddienst) arbeitet und
mit folgendem Kommando aus dem Internet heruntergeladen und installiert
werden k�nnte:\\
\cmdPi{curl https://get.pimoroni.com/onoffshim | bash}
\comment{wurde so nicht durchgef�hrt!}

Diese Software war mir aber zu undurchsichtig, so dass ich lieber das
einfachere und (f�r mich) durchschaubare Verfahren des Anwenders 
\textit{cyberghost} aus dem Retropie-Forum 
(\url{https://retropie.org.uk/}) verwendete. Mir ist an dieser Stelle
wichtig zu betonen, dass auch die Software von \textit{Pimoroni}
einwandfrei funktioniert und ebenso verwendet werden k�nnte.

Grunds�tzlich ist der Einschaltvorgang beim {\onoffshim} ein rein
hardwarem��ig umgesetztes Konzept, das keinerlei zus�tzliche Software
ben�tigt. Beim Ausschalten ist eine Software erforderlich, die die
Bet�tigung des Tasters erkennt und das Betriebssystem vor dem Abschalten
der Versorgungsspannung sauber beendet. Die Variante aus dem
Retro\-pie-Forum 
%\begin{smaller}\url{https://retropie.org.uk/forum/topic/15727/tutorial-onoff-shim-exposed-neat-powerswitch-from-pimoroni/}\end{smaller}
besteht aus lediglich zwei Shellskripten:

\begin{table}[!h]
\centering
\renewcommand{\arraystretch}{1.25}
\begin{tabular}{|p{0.35\textwidth}|p{0.2\textwidth}|p{0.36\textwidth}|}
\hline
\textbf{Skript}							&	\textbf{Aufruf durch}	&	\textbf{Speicherort}\\
\hline
\filenam{onoffshim\_switch.sh}			&	\filenam{/etc/rc.local}	&	\filenam{/home/pi}\\
\hline
\filenam{onoffshim\_gpio-shutoff.sh}	&	\software{systemd}		&	\filenam{/lib/systemd/system-shutdown}\\
\hline
\end{tabular}
\vspace{0.5cm}
\caption{Skripte f�r den \onoffshim}
\label{tab:onoffshim_scripts}
\end{table}


\subsection{Installation der Shellskripte f�r den \onoffshim}
Die beiden Shellskripte aus Tabelle \ref{tab:onoffshim_scripts} k�nnen
vom PC aus dem Downloadverzeichnis dieser Dokumentation mit dem
Kommando\\
\cmdPC{scp ./files/onoffshim/*.sh pi@phoniebox1:/home/pi} in das
Homeverzeichnis des {\RPi} kopiert werden.

Das Skript \filenam{onoffshim\_switch.sh} muss auf dem {\RPi} beim
Hochfahren gestartet werden. Dazu kann es im Homeverzeichnis des
Benutzers verbleiben, muss aber mit Ausf�hrrechten versehen werden:\\
\cmdPi{chmod 755 onoffshim\_switch.sh}\\

Das Skript \filenam{onoffshim\_gpio-shutoff.sh} dagegen soll beim
Herunterfahren des {\RPi} �ber \software{systemd} aufgerufen werden.
Dazu muss es mit Rootrechten in das Verzeichnis
\filenam{/lib/systemd/system-shutdown} kopiert werden, wo es von
\software{systemd} gesucht und ausgef�hrt wird:\\
\cmdPi{sudo mv onoffshim\_gpio-shutoff.sh /lib/systemd/system-shutdown}\\
\cmdPi{sudo chown root:root /lib/systemd/system-shutdown/onoffshim\_gpio-shutoff.sh}\\
\cmdPi{sudo chmod 755 /lib/systemd/system-shutdown/onoffshim\_gpio-shutoff.sh}

Nicht zwingend erforderlich, aber zur Bewahrung der �bersicht lohnt es
sich, im Homeverzeichnis einen symbolischen Link zum neuen Speicherort
des Skriptes \filenam{onoffshim\_gpio-shutoff.sh} anzulegen:\\
\cmdPi{ln -s /lib/systemd/system-shutdown/onoffshim\_gpio-shutoff.sh}


\subsection{Eintr�ge in den aufrufenden Dateien vornehmen}
Damit das Skript \filenam{onoffshim\_switch.sh} beim Hochfahren des
{\RPi} gestartet wird, muss es in die Datei \filenam{/etc/rc.local}
eingetragen werden:\\
\cmdPi{sudo nano /etc/rc.local}\\
\editor{\#!/bin/sh -e\\
\#\\
\# rc.local\\
\#\\
\# This script is executed at the end of each multiuser runlevel.\\
\# Make sure that the script will \dq exit 0\dq\ on success or any other\\
\# value on error.\\
\#\\
\# In order to enable or disable this script just change the execution\\
\# bits.\\
\#\\
\# By default this script does nothing.\\
\\
\# Print the IP address\\
\vdots\\
%\_IP=\$(hostname -I) || true\\
%if \[ "\$\_IP" \]\; then\\
%  printf "My IP address is \%s\bs n" "\$\_IP"\\
%fi\\
\\
\textcolor{red}{\# added by schlizb�da:\\
/home/pi/onoffshim\_switch.sh \& \# do it in the background!}\\
\\
exit 0}


\subsection{Funktionsweise}
Das Skript \filenam{onoffshim\_switch.sh} wird beim Hochfahren �ber den
Eintrag in \filenam{/etc/rc.local} im Hintergrund gestartet und l�uft
solange in einer Schleife, bis der Taster des {\onoffshim} an GPIO17
(Pin 11 der Stiftleiste) bet�tigt wird. Damit wird die Schleife
verlassen und das Herunterfahren des {\RPi} durch Aufruf des Kommandos
\texttt{poweroff} veranlasst.\\
\software{systemd} f�hrt den {\RPi} herunter. Im laufe dieses Prozesses
werden alle ausf�hrbaren Dateien im Verzeichnis
\filenam{/lib/systemd/system-shutdown} gestartet, in dem sich unser
Shellskript \filenam{onoffshim\_gpio-shutoff.sh} befindet:\\
Zun�chst wird �ber den GPIO27 (Pin 13, Variable \$cut\_pin) das Relais
angesteuert, um den Taster des {\onoffshim} k�nstlich zu trennen (siehe
Kapitel \ref{sect:onoffshim_relay}). Der GPIO17 (Pin 11, Variable
\$led\_pin) wird auf Ausgang umkonfiguriert, um zun�chst in einer
for-Schlei\-fe die rote LED auf dem {\onoffshim} dreimal blinken zu
lassen. Schlie�lich wird der GPIO4 (Pin 7, Variable \$poweroff\_pin) als
Ausgang konfiguriert und auf low geschaltet, damit der {\onoffshim} die
Versorgungsspannung abschaltet.

Siehe auch:\\
\begin{scriptsize}\url{https://forum-raspberrypi.de/forum/thread/45820-phoniebox-2-0-rc7-mpd-spielt-ueber-rfid-karte-gewaehlte-musik-nicht-ab/?postID=413400#post413400}\end{scriptsize}

\input{chapter/\lang/chapter3_03}
% chapter3_04.tex -- de (German)
% configuration of mpd
\newpage
\section{\software{Music Player Daemon (mpd)} einrichten}
\label{sect:mpd}

Der \software{Music Player Daemon (mpd)} ist eine Server-Anwendung zum
Abspielen von Musikdateien in den g�ngigen Audiocodecs wie \textit{mp3,
Ogg Vorbis, FLAC, AAC, Mod} oder \textit{WAV}. Clientsoftware kann den
\software{mpd} �ber dessen Protokoll sowohl lokal als auch �ber das
Netzwerk steuern. Damit ist er ideal f�r die Audiowiedergabe auf der
{\Bezeichnung} geeignet.\\
Ein relativer Sprung innerhalb der gesamten Playlist f�r einen schnellen
Vor- und R�cklauf kann �ber den \software{mpd}-Standardclient
\software{mpc} mit dem Kommando \cmd{mpc seekthrough <sec>} abgesetzt
werden. Ein Vorlauf wird �ber positive Werte \textit{mit f�hrendem
Pluszeichen} abgesetzt, ein R�cklauf mit negativen Werten.

\begin{bclogo}[logo = \bclampe, noborder = true]{Hinweis}
Nur leider ist der mit \cmd{apt-get install} aus der
\os{Raspbian}-Dis\-tri\-bu\-tion installierte \software{mpd} fehlerhaft:
Negative Spr�nge f�r den R�cklauf werden nicht ausgef�hrt!
\smiley{nosmile}\\
Es tritt nur ein kurzer Aussetzer in der aktuell wiedergegebenen
Audiodatei auf.
\end{bclogo}

Dieses Problem wurde im \forum bereits diskutiert:\\
\begin{scriptsize}
\url{https://forum-raspberrypi.de/forum/thread/45381-schneller-vor-ruecklauf-durch-when-held-auf-next-prev-buttons-buttons-beim-strea/?postID=435383#post435383}
\end{scriptsize}

Letztendlich kann der Fehler behoben werden, indem der Quellcode von
der Projektseite \url{https://www.musicpd.org/} heruntergeladen und auf
dem {\RPi} kompiliert wird. Danach muss der \software{mpd} richtig in
\software{systemd} eingebunden werden, so dass er beim Booten von
\os{Raspbian} gestartet wird.

\subsection{Standardpaket des \software{mpd} aus \os{Raspbian} installieren}
\label{sect:apt-get_mpd}
Zun�chst wird der \software{mpd} direkt aus der Distribution
installiert:\\
\cmdPi{sudo apt install mpd mpc}
\comment{inklusive dem Client \software{mpc}}\\
Damit wird der \software{mpd} inklusive der erforderlichen 
Konfigurationsdateien installiert.

Standardm��ig wird der \software{mpd} im Verzeichnis \filenam{/usr/bin}
installiert. Die Programmdatei wird aussagekr�ftig umbenannt und ein
symbolischer Link namens \filenam{mpd} darauf erstellt:\\
\cmdPi{mpd --version | head} \comment{Version ermitteln}\\
\stdout{Music Player Daemon \textcolor{red}{0.21.5} (0.21.5)\\ \vdots}

\cmdPi{cd /usr/bin}\\
\cmdPi{sudo mv mpd mpd\_0.21.5\_raspbian}\\
\cmdPi{sudo ln -s /usr/bin/mpd\_0.21.5\_raspbian mpd}

\begin{bclogo}[logo = \bclampe, noborder = true]{Hinweis}
Es ist nat�rlich m�glich, dass der \textit{seek}-Fehler im
\software{mpd} eines Tages behoben wird. Dann ist die Kompilierung nicht
mehr erforderlich und es geht in Abschnitt \ref{sect:mpd-config}
weiter.\\
Ebenso kann man den folgenden zeitaufw�ndigen und f�r den Anf�nger
durchaus l�stigen Teil zun�chst �berspringen, um die {\Bezeichnung} an
sich einigerma�en z�gig zum Laufen zu bringen. Die Kompilierung des 
\software{mpd} kann dann bei Bedarf immer noch nachgeholt werden.
\end{bclogo}


\subsection{Herunterladen und Kompilieren des \software{mpd}}
\label{sect:mpd_recompile}
Auf der Homepage des \software{mpd}-Projektes ist die aktuell
ver�ffentlichte Programmversion ("`stable"') direkt auf der Hauptseite
unter \url{https://www.musicpd.org/} angegeben, siehe Abbildung
\ref{fig:www-musicpd-org}.

\begin{figure}[h]
\centering
\includegraphics[width=0.85\textwidth]{software/musicpd.png}
\caption{Homepage des \software{mpd}-Projektes mit Versionsangabe (0.21.25)}
\label{fig:www-musicpd-org}
\end{figure}

Diese Versionsnummer (in diesem Beispiel \textcolor{red}{0.21.25}) ist
wichtig f�r den Download des Quellcodes:\\
\cmdPi{cd /home/pi}\comment{Download im Home-Verzeichnis durchf�hren!}\\
\cmdPi{wget https://www.musicpd.org/download/mpd/0.21/mpd-0.21.25.tar.xz}\\
\cmdPi{tar xf mpd-0.21.25.tar.xz}\comment{Quellcode entpacken}\\
\cmdPi{cd mpd-0.21.25}\\

\newpage
Unter
\url{https://www.musicpd.org/doc/html/user.html#compiling-from-source}
ist der eigentliche Kompiliervorgang beschrieben. Hier nur die
notwendigen Kommandos:\\
\cmdPi{echo apt install meson g++ \textbackslash\\
 libpcre3-dev libmad0-dev libmpg123-dev libid3tag0-dev \textbackslash\\
 libflac-dev libvorbis-dev libopus-dev libogg-dev \textbackslash\\
 libadplug-dev libaudiofile-dev libsndfile1-dev libfaad-dev \textbackslash\\
 libfluidsynth-dev libgme-dev libmikmod-dev libmodplug-dev \textbackslash\\
 libmpcdec-dev libwavpack-dev libwildmidi-dev \textbackslash\\
 libsidplay2-dev libsidutils-dev libresid-builder-dev \textbackslash\\
 libavcodec-dev libavformat-dev \textbackslash\\
 libmp3lame-dev libtwolame-dev libshine-dev \textbackslash\\
 libsamplerate0-dev libsoxr-dev libbz2-dev \textbackslash\\
 libcdio-paranoia-dev libiso9660-dev libmms-dev \textbackslash\\
 libzzip-dev libcurl4-gnutls-dev libyajl-dev libexpat-dev \textbackslash\\
 libasound2-dev libao-dev libjack-jackd2-dev libopenal-dev \textbackslash\\
 libpulse-dev libshout3-dev libsndio-dev \textbackslash\\
 libmpdclient-dev libnfs-dev libsmbclient-dev \textbackslash\\
 libupnp-dev libavahi-client-dev libsqlite3-dev \textbackslash\\
 libsystemd-dev libgtest-dev libboost-dev libicu-dev \textbackslash\\
 libchromaprint-dev libgcrypt20-dev}
 \comment{alle ben�tigten Bibliotheken installieren}
 
Konfiguration des Quellcode-Projektes f�r den Kompilationsvorgang:\\
\cmdPi{meson . output/release -{}-buildtype=debugoptimized -Db\_ndebug=true}\\
\cmdPi{meson configure output/release}\comment{Anzeige der Compileroptionen}\\
\cmdPi{ninja -C output/release}\comment{Kompiliervorgang starten}

Die Kompilierung besteht aus ca. 632 einzelnen Modulen. Entsprechend
dauert dieser Vorgang auf einem {\RPi} 3B ca. 15\,--\,20 Minuten. Er
l�uft aber erstaunlich unproblematisch durch\dots \ \smiley{smile}

Bewusst weggelassen wird das folgende Kommando, das den soeben
kompilierten \software{mpd} unter \filenam{/usr/local/bin} installieren
w�rde. Der {\autor} hat keine Ahnung, wie er den neuen \software{mpd} an
diesem neuen Pfad (\filenam{/usr/local/bin} statt \filenam{/usr/bin}) so
in das Betriebssystem einbinden muss, dass die restlichen Programme gar
nicht "`merken"', dass hier etwas \textit{gefaked} wurde.
\smiley{sarcastic}\\
\cmdPi{\# sudo ninja -C output/release install}
\comment{nicht ausgef�hrt!}


\subsection{Installieren des \software{mpd}}
Nun muss der neu kompilierte \software{mpd} so installiert werden, dass
er beim Booten vom Betriebssystem geladen und gestartet wird. Dies
geschieht mittlerweile in den meisten GNU/Linux-Distributionen, so auch
in \os{Raspbian} mittels des Init-Systems \software{systemd}.

Zun�chst wird das neu erstellte Kompilat nach \filenam{/usr/bin}
kopiert und der bereits bestehende symbolische Link aus Kapitel
\ref{sect:apt-get_mpd} auf die neue Binary "`umgebogen"':\\
\cmdPi{cd /usr/bin}\\
\cmdPi{\begin{smaller}sudo cp /home/pi/mpd-0.21.25/output/release/mpd /usr/bin/mpd-0.21.25\_compiled\_by\_schlizbaeda\end{smaller}}\\
%\cmdPi{sudo rm /usr/bin/mpd}\comment{Evtl. muss der bestehende Link gel�scht werden}\\
\cmdPi{sudo ln -s /usr/bin/mpd-0.21.25\_compiled\_by\_schlizbaeda \ \ mpd}\\

\begin{bclogo}[arrondi = 0.2, logo = \bcinfo, ombre = true, epOmbre = 0.25, couleurOmbre = black!30,blur]{Achtung}
Leider reicht das blo�e R�berkopieren der neuen Version von
\software{mpd} nach \filenam{/usr/bin} nicht aus, um die alte Variante
1:1 durch die neue Variante zu ersetzen.
\end{bclogo}

% \software{mpd} wird beim Booten
%der {\Bezeichnung} �ber \software{systemd} gestartet. Die Datei 
%\filenam{/etc/systemd/system/mpd.service} muss angepasst werden.

Der Hintergrunddienst \software{mpd} wird beim Hochfahren der
{\Bezeichnung} automatisch �ber \software{systemd} gestartet. Allerdings
schl�gt der Start des neu kompilierten mpd fehl. Nun w�re interessant,
warum es nicht funktioniert. Dies findet man mit folgendem Kommando
heraus:\\
\cmdPi{systemctl status mpd.service}\\
\stdout{\textcolor{red}{\#} mpd.service - Music Player Daemon\\
\_\ \ Loaded: loaded (/etc/systemd/system/mpd.service; enabled; vendor preset: enabled)\\
\_\ \ Active: \textcolor{red}{failed} (Result: exit-code) since Mon 2020-08-24 21:02:39 CEST; 13s ago\\
\_\ \ \ \ Docs: man:mpd(1)\\
\_\ \ \ \ \ \ \ \ \ \ man:mpd.conf(5)\\
\_\ \ \ \ \ \ \ \ \ \ file:///usr/share/doc/mpd/user-manual.html\\
\_\ Process: 6483 ExecStart=/usr/bin/mpd --no-daemon \$MPDCONF \textcolor{red}{(code=exited, status=1/FAILURE)}\\
\_Main PID: 6483 (code=exited, status=1/FAILURE)\\
\\
Aug 24 21:02:38 phoniebox1 systemd[1]: Starting Music Player Daemon...\\
Aug 24 21:02:39 phoniebox1 mpd[6483]: exception: No configuration file found\\
Aug 24 21:02:39 phoniebox1 systemd[1]: mpd.service: Main process exited, code=exited, status=1/FAILURE\\
Aug 24 21:02:39 phoniebox1 systemd[1]: mpd.service: Failed with result 'exit-code'.\\
Aug 24 21:02:39 phoniebox1 systemd[1]: \textcolor{red}{Failed to start Music Player Daemon.}
}

\software{systemd} erkannte, dass das Aufrufkommando \cmd{/usr/bin/mpd 
--no-daemon \$MPDCONF} mit einem fehlerhaften
Exitcode ungleich 0 beendet wurde. Im neu kompilierten \software{mpd}
trat also ein Fehler auf. Letzlich liegt es offenbar daran, dass mit dem
neuen \software{mpd} auf die Shellvariable \cmd{\$MPDCONF} irgendwie
nicht richtig zugegriffen werden kann. 
\begin{bclogo}[logo = \bclampe, noborder = true]{Jamileckstam�rsch!}
Das muss der Linux-Dauernoob {\autor} nicht verstehen!
\end{bclogo}
Der \textit{quick+dirty}-Ansatz vom {\autor} besteht darin, die
Servicedatei von \software{systemd} so anzupassen, dass der Pfad der
Konfigdatei \filenam{/etc/mpd.conf} nicht �ber die omin�se Shellvariable
\cmd{\$MPDCONF}, sondern direkt angegeben wird. Dazu wird die
Servicedatei \filenam{/etc/systemd/system/mpd.service} angepasst:\\
\cmdPi{sudo systemctl edit --full mpd.service}
%\comment{offizielle systemd-Anpassung: startet den Standardeditor (nano)}\\
\comment{offizielle Bearbeitung der systemd-Servicedatei}\\
\editor{[Service]\\
        Type=notify\\
        EnvironmentFile=/etc/default/mpd\\
        \textcolor{red}{\#}ExecStart=/usr/bin/mpd --no-daemon \$MPDCONF \textcolor{red}{\# auskommentiert!\\
        \# adjusted by schlizb�da at 2020-08-16:\\
        ExecStart=/usr/bin/mpd --no-daemon /etc/mpd.conf}
       }

\cmdPi{sudo systemctl restart mpd.service}
\comment{mpd neu starten}


\subsection{Res�mee der Kompilierung von \software{mpd} -- es w�re einfacher gewesen\dots}
Das im \software{mpd}-Projekt enthaltene
\software{make}-Installationsskript\\
\cmd{sudo ninja -C output/release install}\\
installiert die kompilierte Bin�rdatei nach \filenam{/usr/local/bin} und
nicht nach \filenam{/usr/bin}. Dies ist aber -- entgegen meiner ersten
Einsch�tzung -- kein Problem, da der Start des \software{mpd} �ber das
Init-System \software{systemd} erfolgt. Alle Programmaufrufe, die �ber
\software{systemd} gestartet werden, sind mit absolutem Pfad in den
\software{systemd}-Steuerdateien, den sogenannten Units hinterlegt. Dort
muss man "`nur"' in der f�r \software{mpd} zust�ndigen Unit
\filenam{/etc/systemd/system/mpd.service} den Aufrufpfad
\filenam{/usr\textcolor{red}{/local}/bin} im Eintrag \cmd{ExecStart=...}
setzen. Das im vorigen Kapitel beschiebene Umkopieren der
Originalversion samt symbolischen Link auf die neu kompilierte Version
h�tte es \textit{vermutlich} gar nicht gebraucht.

Leider habe ich das erst sp�ter gemerkt! Aber ich musste dann ohnehin an
die Unit ran, denn meine neu kompilierte Variante des \software{mpd}
kann nicht auf die Shellvariable \cmd{\$MPDCONF} zugreifen.

Diesen Sachverhalt habe ich im deutschen{\forum} beschrieben unter\\
\begin{scriptsize}
\url{https://forum-raspberrypi.de/forum/thread/45381-schneller-vor-ruecklauf-durch-when-held-auf-next-prev-buttons-buttons-beim-strea/?postID=446062#post446062}
\end{scriptsize}

Das Init-System \software{systemd} ist unter
\url{https://ubuntuusers.de/} recht gut beschrieben:

Allgemeine Erkl�rung: \url{https://wiki.ubuntuusers.de/systemd/}\\
Steuerdateien (Units): \url{https://wiki.ubuntuusers.de/systemd/Units}\\
Konfiguration mit \cmd{systemctl}: \url{https://wiki.ubuntuusers.de/systemd/systemctl/}

\begin{bclogo}[logo = \bclampe, noborder = true]{Hinweis zu einem echten Problem von FLOSS}
Dies ist ein Beispiel f�r ein markantes und ernstzunehmendes Problem von
FLOSS (Free/Libre Open Source Software): So behaupten viele Bef�rworter
von FLOSS (auch der \autor), dass nur mit quelloffenem Code verhindert
werden k�nne, fehlerhaften Code oder gar Schadcode auf ein System
einzuspielen. Oder zumindest, dass dies nachvollziehbar sei.\\
Allerdings tut die aus dem Raspbian-Repository installierte
Programmversion von \software{mpd} nicht das, was im heruntergeladenen
Quellcode drinsteht. Der \software{mpd} f�hrt keinen R�cklauf aus. Aber
wenn der heruntergeladene Code kompiliert wird, tut er es pl�tzlich
doch. Es stellt sich unweigerlich die Frage, welcher Code denn nun f�r
die Bin�rdatei im Raspbian-Repository verwendet wurde und was der
mit dem von mir heruntergeladenen Code zu tun hat (wenngleich dies beim
\software{mpd} keine Absicht ist).\\
Die obige Argumentation, dass ein System genau das macht, was in
irgendwelchen ver�ffentlichten Quellcodedateien drinsteht, gilt
eigentlich nur dann, wenn man das System auch wirklich selbst
kompiliert!\\
FLOSS: \url{https://de.wikipedia.org/wiki/Free/Libre_Open_Source_Software}
\end{bclogo}


\subsection{Konfiguration des \software{mpd}}
\label{sect:mpd-config}
Die gesamte Konfiguration des \software{mpd} wird in der Datei
\filenam{/etc/mpd.conf} vorgenommen. Diese Datei besteht aus Eintr�gen
in der Form \textit{Schl�sselwort Wert} und in der Standardauslieferung
aus vielen Kommentarzeilen. Letztlich m�ssen folgende Eintr�ge erg�nzt
oder angepasst werden:\\
\cmdPi{sudo nano /etc/mpd.conf}\\
\editor{music\_directory    \dq/home/pi/RPi-Jukebox-RFID/shared/audiofolders\dq\\
playlist\_directory \dq/home/pi/RPi-Jukebox-RFID/playlists\dq\\
user \ \ \ \ \ \ \ \ \ \ \ \ \ \ \dq root\dq\\
auto\_update \ \ \ \ \ \ \ \dq yes\dq\\ % (you have to remove the \# in front of that line)\\
auto\_update\_depth \ \dq 10\dq\\ % (remove the \# and change the value to 10)\\
%\#mixer\_control \ \ \ \ \ \dq yourAudioIfaceNameHere\dq\\ % (you need to uncomment this line and change the audio iFace shortname)\\
mixer\_control \ \ \ \ \ \dq Master\dq
}
        
Folgendes Kommando dient dazu, die Bezeichnung f�r den Eintrag
\texttt{mixer\_control} herauszufinden:\\
\cmdPi{amixer scontrols}\comment{liefert z.B. bei \miniamp-Installation}\\
\stdout{Simple mixer control 'Master',0}
\os
Der \software{mpd} wird mit der neuen Konfiguration mit folgendem
Kommando gestartet:\\
\cmdPi{mpc update}

\textbf{Test von \software{mpd}}\\
Jetzt ist der Zeitpunkt gekommen, einen geschmeidigen Audiotest mit
\software{mpd} durchzuf�hren. Dazu zun�chst vom PC eine Audiodatei
auf den {\RPi} in das oben konfigurierte Audioverzeichnis
\filenam{/home/pi/RPi-Jukebox-RFID/shared/audiofolders} kopieren und
mit dem Client \software{mpc} abspielen:\\
\cmdPC{scp /Pfad/Musik.flac pi@phoniebox1:/home/pi/RPi-Jukebox-RFID/shared/audiofolders}

\cmdPi{mpc add /home/pi/RPi-Jukebox-RFID/shared/audiofolders/Musik.flac}\\
\cmdPi{mpc play}

Sollte keine Musik h�rbar sein, mit dem \software{alsamixer} die
Lautst�rke �berpr�fen und anpassen:\\
\cmdPi{alsamixer}

(siehe Abbildung \ref{fig:alsamixer})

% chapter3_05.tex -- de (German)
% configuration of mpd
\newpage
\section{\software{Music Player Daemon (mpd)} einrichten}
Die gesamte Konfiguration des \software{mpd} wird in der Datei
\filenam{/etc/mpd.conf} vorgenommen. Diese Datei besteht aus Eintr�gen
in der Form \textit{Schl�sselwort Wert} und in der Standardauslieferung
aus vielen Kommentarzeilen. Letztlich m�ssen folgende Eintr�ge erg�nzt
oder angepasst werden:\\
\cmdPi{sudo nano /etc/mpd.conf}\\
\editor{music\_directory    "{}/home/pi/RPi-Jukebox-RFID/shared/audiofolders"\\
playlist\_directory "{}/home/pi/RPi-Jukebox-RFID/playlists"\\
user               "root"\\
auto\_update        "yes"\\ % (you have to remove the \# in front of that line)\\
auto\_update\_depth  "10"\\ % (remove the \# and change the value to 10)\\
%\#mixer\_control      "yourAudioIfaceNameHere"\\ % (you need to uncomment this line and change the audio iFace shortname)\\
mixer\_control      "Master"
}
        
Folgendes Kommando dient dazu, die Bezeichnung f�r den Eintrag
\texttt{mixer\_control} herauszufinden:\\
\cmdPi{amixer scontrols}\comment{liefert z.B. bei \miniamp-Installation}\\
\stdout{Simple mixer control 'Master',0}

Der \software{mpd} wird mit der neuen Konfiguration mit folgendem
Kommando gestartet:\\
\cmdPi{mpc update}

\textbf{Test von \software{mpd}}\\
Jetzt ist der Zeitpunkt gekommen, einen geschmeidigen Audiotest mit
\software{mpd} durchzuf�hren. Dazu zun�chst vom PC eine Audiodatei
auf den {\RPi} in das oben konfigurierte Audioverzeichnis
\filenam{/home/pi/RPi-Jukebox-RFID/shared/audiofolders} kopieren und
mit dem Client \software{mpc} abspielen:\\
\cmdPC{scp /Pfad/Musik.flac pi@phoniebox1:/home/pi/RPi-Jukebox-RFID/shared/audiofolders}

\cmdPi{mpc add /home/pi/RPi-Jukebox-RFID/shared/audiofolders/Musik.flac}\\
\cmdPi{mpc play}

Sollte keine Musik h�rbar sein, mit dem \software{alsamixer} die
Lautst�rke �berpr�fen und anpassen:\\
\cmdPi{alsamixer}

(siehe Abbildung \ref{fig:alsamixer})


% chapter3_06.tex -- de (German)
% installation of the RFID reader
\newpage
%\begin{bclogo}[logo = \bclampe, noborder = true]{Hinweis}
\begin{bclogo}[arrondi = 0.2, logo = \bcinfo, ombre = true, epOmbre = 0.25, couleurOmbre = black!30,blur]{Achtung}
Ab hier bezieht sich diese Dokumentation auf die Seite\\
\url{https://github.com/MiczFlor/RPi-Jukebox-RFID/wiki/CONFIGURE-stretch}\\
von MiczFlor!
\end{bclogo}

\section{{\reader} installieren}
In diesem Abschnitt wird die Installation des \textit{Neuftech USB
RFID-Reader 125kHz} beschrieben, der in {\autor}s {\Bezeichnung}
verbaut wurde und bei amazon erh�ltlich ist:\\
\url{https://www.amazon.de/Neuftech-Reader-Kartenleseger%C3%A4t-Kartenleser-Kontaktlos/dp/B018OYOR3E}

Dieses Ger�t emuliert am USB-Anschluss eine Tastatur (ein Ger�t aus der
USB-Klasse \textit{Human Interface Device},
\url{https://de.wikipedia.org/wiki/Human_Interface_Device}).\\
Die \Bezeichnung-Soft\-ware f�ngt Ereignisse vom {\reader} jedoch �ber
\software{udev} ab:\\
\url{https://wiki.ubuntuusers.de/udev/}


\subsection{{\reader} �berpr�fen}
Einen unter \os{Raspbian} korrekt installierten {\reader} erkennt man
durch Kontrolle folgender Punkte
\begin{compactitem}
\item{Die LED des {\reader}s leuchtet -- im eingebauten Zustand evtl. nicht erkennbar}
\item{Piepton beim Hochfahren}
\item{Piepton beim Lesen einer \Karte}
\item{�berpr�fen, ob das Softwareevent f�r den {\reader} ordnungsgem��
      eingerichtet ist:\\
      \cmdPi{ls -la /dev/input/by-id}\\
      \stdout{total 0\\
              drwxr-xr-x 2 root root  60 Jan 11 15:36 .\\
              drwxr-xr-x 4 root root 120 Jan 11 15:36 ..\\
              lrwxrwxrwx 1 root root   9 Jan 11 15:36 usb-HXGCoLtd\_27db-event-kbd -> ../event0}
     }
\end{compactitem}


\subsection{{\reader} in der \Bezeichnung-Software registrieren}
\cmdPi{cd /home/pi/RPi-Jukebox-RFID/scripts}\\
\cmdPi{python3 RegisterDevice.py}\comment{liefert \zB:}\\
\stdout{Choose the reader from list\\
0 HID 046a:0011\\
\textcolor{red}{1 HXGCoLtd Keyboard}\\
Device Number: \textbf{\textcolor{red}{1}}}\\
Unter GNU/Linux wird der Chipsatz des {\reader}s als \textit{HXGCoLtd
Keyboard} erkannt. Daher ist bei der Abfrage der \texttt{Device Number}
in diesem Beispiel der Wert \textcolor{red}{1} einzugeben.

Ob die Registrierung des {\reader}s erfolgreich war, kann mit der Datei\\
\filenam{/home/pi/RPi-Jukebox-RFID/scripts/deviceName.txt} gepr�ft
werden. Diese Datei enth�lt den oben gew�hlten Ger�tenamen:\\
\cmdPi{cat deviceName.txt}\\
\stdout{HXGCoLtd Keyboard}


\subsection{RFID-Konfigurationsdatei der \Bezeichnung-Software anlegen}
\label{sect:RFID_cfg}
\cmdPi{cd /home/pi/RPi-Jukebox-RFID/settings}\\
\cmdPi{cp rfid\_trigger\_play.conf.sample rfid\_trigger\_play.conf}\\
\cmdPi{sudo chown pi:pi rfid\_trigger\_play.conf}\\ % braucht's das?
\cmdPi{sudo chmod 665 rfid\_trigger\_play.conf}     % braucht's das?













% chapter3_07.tex -- de (German)
% audio parameters
\section{Konfigurationsdateien im Verzeichnis \filenam{RPi-Jukebox-RFID/settings}}
Das Unterverzeichnis filenam{/home/pi/RPi-Jukebox-RFID/settings} enth�lt
einige Konfigurationsdateien, in denen das Verhalten der {\Bezeichnung}
eingestellt und angepasst werden kann. Der Dateiinhalt bestaht dabei nur
aus dem gew�nschten Parameter. Die Funktion wird durch den Dateinamen
bestimmt und ist in den jeweiligen Python- \bzw Shellskripten fest
hinterlegt, siehe\\
\begin{smaller}\url{https://github.com/MiczFlor/RPi-Jukebox-RFID/wiki/CONFIGURE-stretch#create-settings-for-audio-playout}\end{smaller}

\subsection{Audioeinstellungen}
Die folgenden Kommandos legen im Verzeichnis
\filenam{/home/pi/RPi-Jukebox-RFID/settings} kleine Textdateien an, die
das Verhalten der Taster \textit{volume up} und \textit{volume down}
konfigurieren. Ferner kann die maximale Lautst�rke begenzt werden:\\
\cmdPi{echo \dq Master\dq\ > /home/pi/RPi-Jukebox-RFID/settings/Audio\_iFace\_Name}\\
\cmdPi{echo \dq 3\dq\ > /home/pi/RPi-Jukebox-RFID/settings/Audio\_Volume\_Change\_Step}\\
\cmdPi{echo \dq 100\dq\ > /home/pi/RPi-Jukebox-RFID/settings/Max\_Volume\_Limit}

MP3-Dateien f�r Startup und Shutdown kopieren:\\
\cmdPi{\begin{scriptsize}cp /home/pi/RPi-Jukebox-RFID/misc/sampleconfigs/startupsound.mp3.sample\\ /home/pi/RPi-Jukebox-RFID/shared/startupsound.mp3\end{scriptsize}}\\
\cmdPi{\begin{scriptsize}cp /home/pi/RPi-Jukebox-RFID/misc/sampleconfigs/shutdownsound.mp3.sample\\ /home/pi/RPi-Jukebox-RFID/shared/shutdownsound.mp3\end{scriptsize}}

\subsection{automatische Abschaltung bei Nichtbenutzung}
Gerade bei Kindern ist es sinnvoll, die {\Bezeichnung} nach einiger Zeit
der Nichtverwendung automatisch abzuschalten, um die Akkulaufzeit zu
verl�ngern. Der angegebene Wert ist inaktive Zeit in Minuten
\textit{ab dem Beenden der \software{mpd}-Playlist}, bevor sich die Box
automatisch abschaltet. Der Wert 0 deaktiviert diese Funktion.\\ 
\cmdPi{echo \dq 0\dq\ > /home/pi/RPi-Jukebox-RFID/settings/Idle\_Time\_Before\_Shutdown}

Ein automatisches Abschalten der {\Bezeichnung} nach 15 Minuten
Nichtbenutzung erreicht man alternativ durch:\\
\cmdPi{echo \dq 15\dq\ > /home/pi/RPi-Jukebox-RFID/settings/Idle\_Time\_Before\_Shutdown}%\comment{Alternative}

\subsection{Was soll beim wiederholten Auflegen einer {\Karte} passieren?}
Quelle:\\
\url{https://github.com/MiczFlor/RPi-Jukebox-RFID/wiki/MANUAL#second-swipe}

In der Datei \filenam{/home/pi/RPi-Jukebox-RFID/settings/Second\_Swipe}
wird festgelegt, was beim wiederholten Auflegen derselben {\Karte}
passieren soll. Dazu wird in diese Datei das entsprechende Schl�sselwort
aus folgender Liste eingetragen:

\begin{table}[h]
\centering
\renewcommand{\arraystretch}{1.25}
\begin{tabular}{|p{0.15\textwidth}|p{0.80\textwidth}|}
\hline
\textbf{Schl�sselwort}	&	\textbf{Funktion}\\
\hline
\texttt{RESTART}		&	Neustart der aktiven Playlist von Anfang an\\
\hline
\texttt{PAUSE}			&	Umschalten zwischen \textit{Play/Pause}\\
\hline
\texttt{SKIPNEXT}		&	Zum n�chsten Titel aus der Playlist springen\\
\hline
\texttt{NOAUDIOPLAY}	&	Keine Beeinflussung der Audiowiedergabe\\
\hline
\end{tabular}
\vspace{0.5cm}
\caption{Verhalten bei wiederholtem Auflegen derselben \Karte}
\label{tab:second_swipe}
\end{table}

Bei meiner Box habe ich mich f�r den Eintrag \texttt{PAUSE}
entschieden.\\
\cmdPi{nano /home/pi/RPi-Jukebox-RFID/settings/Second\_Swipe}\\
\stdout{PAUSE}

\begin{bclogo}[logo = \bclampe, noborder = true]{Hinweis}
In der Webapp ist die Einstellung sicherer...
\end{bclogo}

\begin{figure}[h]
\centering
\includegraphics[width=0.90\textwidth]{/software/Second_Swipe.png}
\caption{Einstellung des Verhaltens bei wiederholtem Auflegen derselben {\Karte} �ber die WebApp}
\end{figure}

% chapter3_08.tex -- de (German)
% Using GPIO hardware buttons
\section{Die {\Bezeichnung} mit Tastern an der GPIO-Leiste steuern}
\label{sect:GPIO-buttons}

Urspr�nglich war vorgesehen, bestimmte {\Karte}n mit
Steuerungsfunktionen zu belegen. Dieses Konzept ist aber gerade bei
Kindern ung�nstig, da (zumindest meine) Kinder solche Sachen in den
Tiefen ihrer Zimmer sehr gut verstecken \smiley{smile} und die
{\Bezeichnung} damit zwischenzeitlich(?) unbedienbar wird.\\ 
Daher hat MiczFlor in seinem github-Repository das Kapitel
\url{https://github.com/MiczFlor/RPi-Jukebox-RFID/wiki/Using-GPIO-hardware-buttons}
eingef�gt, in dem die Inbetriebnahme der Steuerung �ber Taster an der
GPIO-Leiste beschrieben wird.

\begin{bclogo}[arrondi = 0.2, logo = \bcinfo, ombre = true, epOmbre = 0.25, couleurOmbre = black!30,blur]{Achtung}
Die Pinbelegung in der Anleitung von MiczFlor ordnet den Tastern Pins
zu, die f�r die digitale Audio�bertragung �ber I2S zum HifiBerry
{\miniamp} ben�tigt werden! Daher muss das Pythonskript f�r die
Abfrage der Taster stark angepasst werden!
\end{bclogo}

F�r den Zugriff auf die GPIOs wird die Python-Bibliothek
\software{gpiozero} verwendet, von der es jeweils eine eigene Variante
f�r Python2 und Python 3 gibt. Installation der Bibliothek �ber
folgendes Kommando:\\
\cmdPi{sudo apt-get install python3-gpiozero python-gpiozero}
\comment{f�r Python2 \textbf{und} 3!}

Zur Abfrage der Taster wird das Pythonskript
\filenam{/home/pi/RPi-Jukebox-RFID/scripts/gpio-buttons.py} verwendet.
Die Originalvorlage k�nnte mit folgendem Kommando installiert werden:\\
\cmdPi{\begin{scriptsize}sudo cp /home/pi/RPi-Jukebox-RFID/misc/sampleconfigs/gpio-buttons.py.sample\\ /home/pi/RPi-Jukebox-RFID/scripts/gpio-buttons.py\end{scriptsize}}
\comment{funktioniert nicht mit dem \miniamp!}\\
Allerdings kollidiert die Pinvergabe mit der Pinzuordnung des HifiBerry
{\miniamp}s!

Bei Einbau des {\miniamp}s muss die von {\autor} angepasste Datei
verwendet werden! Diese Datei entspricht der Pinbelegung der
Lochrasterplatine aus Kapitel \ref{sect:prototypingboard}:\\
\cmdPC{scp ./files/GPIO/gpio-buttons.py pi@phoniebox1:/home/pi/RPi-Jukebox-RFID/scripts}

\cmdPi{nano /home/pi/RPi-Jukebox-RFID/scripts/gpio-buttons.py}\\
%\begin{verbatim}
\verb|#shut = Button(3,hold_time=2)  # --> Pin 5 # no longer necessary due to OnOffShim|\\
\verb|vol0 = Button(13,pull_up=True) # --> Pin 33|\\
\verb|volU = Button(12,pull_up=True,hold_time=0.3,hold_repeat=True) # --> Pin 32|\\
\verb|volD = Button(6,pull_up=True,hold_time=0.3,hold_repeat=True)  # --> Pin 31|\\
\verb|next = Button(7,pull_up=True)  # --> Pin 26|\\
\verb|prev = Button(8,pull_up=True)  # --> Pin 24|\\
\verb|halt = Button(5,pull_up=True)  # --> Pin 29|\\
\verb|#reco = Button(6, pull_up=True) # Choose GPIO to fit your hardware|\\
\verb|#play = Button(12,pull_up=True) # Choose GPIO to fit your hardware|
%\end{verbatim}

siehe auch Tabelle \ref{tab:gpio_rpi}

% chapter3_9.tex -- de (German)
% assigning RFID cards to audio files/folders
\section{Zuordnung von H�rspielen zu {\Karte}n }
\label{sect:assignment}

\begin{bclogo}[logo = \bclampe, noborder = true]{Hinweis}
Die funktionalen Elemente der {\Bezeichnung} sind nun installiert. Jetzt
ist es an der Zeit, einzelnen H�rspielen oder Musikst�cken {\Karte}n
zuzuordnen und einen \textit{Soundcheck} zu machen, um die grundlegende
Funktionalit�t der Box zu �berpr�fen:
\end{bclogo}

\begin{figure}[!h]
\centering
\includegraphics[width=0.45\textwidth]{/software/helloween.jpg}
\includegraphics[width=0.45\textwidth]{/software/metallica.jpg}
\caption{Audiodaten (Musikalben) auf die {\Bezeichnung} kopieren}
\end{figure}

Zun�chst werden \textit{rekursiv} zwei Unterverzeichnisse mit den
Musikdateien zweier Musikalben vom PC in den Audioordner auf der 
{\Bezeichnung} kopiert:\\
\cmdPC{scp -r Helloween pi@phoniebox1:/home/pi/RPi-Jukebox-RFID/shared/audiofolders}\\
\cmdPC{scp -r Metallica pi@phoniebox1:/home/pi/RPi-Jukebox-RFID/shared/audiofolders}

Kontrolle mit \cmd{ls -lR} auf der {\Bezeichnung}:\\
\cmdPi{ls -lR /home/pi/RPi-Jukebox-RFID/shared/audiofolders}
\begin{smaller}
\begin{verbatim}
total 26728
drwxr-xr-x 2 pi pi           4096 May 17 15:50  Helloween
drwxr-xr-x 2 pi pi           4096 May 17 15:56  Metallica

./Helloween:
total 104140
-rwxr-xr-x 1 pi pi 37314544 May 17 15:47 01_I_Want_Out.flac
-rwxr-xr-x 1 pi pi 38112711 May 17 15:48 02_Dr._Stein.flac
-rwxr-xr-x 1 pi pi 31203347 May 17 15:49 03_Future_World.flac

./Metallica:
total 151828
-rwxr-xr-x 1 pi pi 38978523 May 17 15:52 01_Enter_Sandman.flac
-rwxr-xr-x 1 pi pi 41647915 May 17 15:54 02_Sad_But_True.flac
-rwxr-xr-x 1 pi pi 29319722 May 17 15:54 03_Holier_Than_Thou.flac
-rwxr-xr-x 1 pi pi 45518619 May 17 15:56 04_The_Unforgiven.flac
\end{verbatim}
\end{smaller}

\subsection{{\Karte} manuell zuordnen}
Um nun diesen beiden Alben zwei neue {\Karte}n zuzuordnen, sind folgende
Schritte erforderlich:
\begin{compactitem}
\item{ID der neuen {\Karte} herausfinden}
\item{F�r jede ID die Shortcut-Datei nacheditieren.} 
%\item{Playlist als \cmd{*.m3u}-Datei erstellen} # braucht's nicht!
%%%%
%\item{F�r jede ID die Shortcut-Datei \filenam{/home/pi/RPi-Jukebox-RFID/shared/shortcuts/000???????} anlegen, in der der Name einer \cmd{*.m3u}-Datei eingetragen ist.} 
%\item{Im Verzeichnis \filenam{/home/pi/RPi-Jukebox-RFID/playlists} die \cmd{*.m3u}-Datei mit dem oben angegebenen Namen anlegen}
\end{compactitem}

Zun�chst wird eine neue {\Karte} �ber den {\reader} eingelesen. Dabei
wird im Unterverzeichnis \filenam{/home/pi/RPi-Jukebox-RFID/shared/shortcuts}
eine Datei angelegt, die den Namen der Karten-ID hat, \zB \cmd{0009563230}.
Der Inhalt dieser Datei muss auf den Verzeichnisnamen des Albums oder
H�rbuchs unter \filenam{/home/pi/RPi-Jukebox-RFID/shared/audiofolders}
angepasst werden, \zB "`Helloween"'.\\
\cmdPi{nano /home/pi/RPi-Jukebox-RFID/shared/shortcuts/0009563230}\\
\stdout{Helloween}

%\subsection{ID einer neuen {\Karte} ermitteln}
%Die ID einer {\Karte} wird beim Einlesen durch den {\reader} in der
%Datei \filenam{/home/pi/RPi-Jukebox-RFID/settings/Latest\_RFID} abgelegt.
%Nach erfolgtem Einlesen den Inhalt dieser Datei anzeigen:\\
%\cmdPi{cat /home/pi/RPi-Jukebox-RFID/settings/Latest\_RFID}\\
%\stdout{0009563230}
%
%\subsection{Shortcut-Datei f�r die soeben ermittelte ID anlegen}
%Im Unterverzeichnis \filenam{/home/pi/RPi-Jukebox-RFID/shared/shortcuts}
%wird eine Datei angelegt, deren Dateiname der ermittelten ID, \zB 
%\cmd{0009563230} entspricht. In dieser Datei wird der Name des
%Unterverzeichnisses des gew�nschten Albums in\\
%\filenam{/home/pi/RPi-Jukebox-RFID/shared/audiofolders} eingetragen, \zB
%"`Helloween"'.\\
%\cmdPi{nano /home/pi/RPi-Jukebox-RFID/shared/shortcuts/0009563230}\\
%\stdout{Helloween}
%
%\subsection{Playlist f�r dieses Album als \cmd{*.m3u}-Datei erstellen}
%Die Playlist, die bei Auflegen dieser Karte abgespielt werden soll,
%muss im Verzeichnis\\ \filenam{/home/pi/RPi-Jukebox-RFID/playlists} als
%\cmd{*.m3u}-Datei erstellt werden. Dabei handelt es sich einfachsten
%Fall um eine Textdatei, die in jeder Zeile eine abzuspielende Audiodatei
%enth�lt, siehe \url{https://de.wikipedia.org/wiki/M3U}:\\
%\cmdPi{nano /home/pi/RPi-Jukebox-RFID/playlists/Helloween.m3u}\\
%\stdout{/home/pi/RPi-Jukebox-RFID/shared/audiofolders/Helloween/01\_I\_Want\_Out.flac\\
%        /home/pi/RPi-Jukebox-RFID/shared/audiofolders/Helloween/02\_Dr.\_Stein.flac\\
%        /home/pi/RPi-Jukebox-RFID/shared/audiofolders/Helloween/03\_Future\_World.flac
%       }

Nach erneutem Auflegen der {\Karte} mit der ID \cmd{0009563230} werden
%die in der Playlist hinterlegten Musiktitel des Albums \textit{Helloween
%-- The Best, The Rest, The Rare} abgespielt.
die im Verzeichnis \filenam{/home/pi/RPi-Jukebox-RFID/shared/audiofolders/Helloween}
abgelegten Audiodateien abgespielt.

%\begin{bclogo}[logo = \bclampe, noborder = true]{Hinweis}
%Wichtig ist bei diesem Vorgehen, dass sowohl der Dateiname der
%Playlist als auch das Albumsverzeichnis genau der Bezeichnung aus der
%Shortcut-Datei entsprechen muss, hier also "`Helloween"'!
%\end{bclogo}

\subsection{{\Karte} �ber die Webanwendung registrieren}
Alternativ kann eine {\Karte} auch mit Hilfe des Webservices zugeordnet
werden. Das Vorgehen wird in der folgenden Bilderstrecke dargestellt:

\begin{figure}[!h]
\centering
\includegraphics[width=0.49\textwidth]{/software/webservice-metallica01.png}
\includegraphics[width=0.49\textwidth]{/software/webservice-metallica02.png}
\includegraphics[width=0.49\textwidth]{/software/webservice-metallica03.png}
\includegraphics[width=0.49\textwidth]{/software/webservice-metallica04.png}
\caption{{\Karte} �ber Webservice registrieren}
\end{figure}

Am PC wird �ber einen Internetbrowser (Firefox) eine Verbindung zur
{\Bezeichnung} hergestellt, z.B. durch Eingabe der IP-Adresse (hier
192.168.178.154). Durch Klick auf \menuitem{Settings} in der oberen
Men�leiste erscheint das Bild links oben. Hier muss auf die Schaltfl�che
\button{Register new card ID} geklickt werden. Nun erscheint das Bild
rechts oben. Jetzt die neue {\Karte} �ber den {\reader} einlesen. Im
dazugeh�rigen Textfeld wird die ID aktualisiert (hier \cmd{0009593627}).

In der Zeile \menuitem{Audio Folder} werden im Auswahlfeld (ComboBox)
alle Unterverzeichnisse von \filenam{/home/pi/RPi-Jukebox-RFID/shared/audiofolders}
angezeigt. Hier den gew�nschten Ordner "`Metallica"' ausw�hlen. Zum
Speichern der gew�hlten Einstellungen im Browser nach unten scrollen
und die gr�ne Schaltfl�che \button{Submit} anklicken.

Eine �berpr�fung im ssh-Terminal zeigt die Zuordnung der neuen {\Karte}
zum Album "`Metallica"':\\
\cmdPi{cat ~/RPi-Jukebox-RFID/shared/shortcuts/0009593627}\\
\stdout{Metallica}

Auch hier bewirkt das erneute Auflegen dieser Karte den Start der
Wiedergabe der Audiodateien.

\subsection{Testen der Taster an den GPIO-Pins}
In der Konfiguration von {\autor}s {\Bezeichnung} sind in der Datei\\
\filenam{/home/pi/RPi-Jukebox-RFID/scripts/gpio-buttons.py} an folgenden
GPIOs Taster definiert, siehe auch Kapitel \ref{sect:prototypingboard}
sowie Tabelle \ref{tab:gpio_rpi}:

\begin{table}[!h]
\centering
\renewcommand{\arraystretch}{1.25}
\begin{tabular}{|p{0.10\textwidth}|p{0.10\textwidth}|p{0.15\textwidth}|p{0.30\textwidth}|p{0.20\textwidth}|}
\hline
\textbf{GPIO}	&	\textbf{Pin}	&	\textbf{Funktion}	&	\textbf{Beschreibung}	&	\textbf{Anmerkung}\\
\hline
\texttt{13}		&	\texttt{33}		&	\textit{vol0}		&	stumm schalten			&	(nicht verwendet)\\
\hline
\texttt{12}		&	\texttt{32}		&	\textit{volU}		&	Lautst�rke erh�hen		&\\
\hline
\texttt{6}		&	\texttt{31}		&	\textit{volD}		&	Lautst�rke verringern	&\\
\hline
\texttt{7}		&	\texttt{26}		&	\textit{next}		&	n�chster Titel			&\\
\hline
\texttt{8}		&	\texttt{24}		&	\textit{prev}		&	vorheriger Titel		&\\
\hline
\texttt{5}		&	\texttt{29}		&	\textit{halt}		&	Play/Pause				&\\
\hline
\end{tabular}
\vspace{0.5cm}
\caption{Taster an der GPIO-Leiste bei {\autor}s \Bezeichnung}
\label{tab:gpio_buttons}
\end{table}

Die Taster k�nnen bei dieser Gelegenheit auch gleich auf Funktion
�berpr�ft werden \smiley{smile}

%    % chapter3_10.tex -- de (German)
%    % auto start
%    \section{\software{systemd:} Autostart der \Bezeichnung-Software einrichten}
%    \label{sect:systemd}
%    Quelle:\\
%    \url{https://github.com/MiczFlor/RPi-Jukebox-RFID/wiki/CONFIGURE-stretch#auto-start-the-phoniebox}
%    
%    \begin{bclogo}[logo = \bclampe, noborder = true]{Hinweis}
%    Die in den folgenden Kommandos angegebenen Dateien wurden zwar f�r
%    \os{Raspbian Stretch} vorgesehen, aber auch dort kam schon
%    \software{systemd} zum Einsatz. Daher ist es unproblematisch, dass hier
%    die \os{Stretch}-Dateien f�r \os{Raspbian Buster} verwendet werden.
%    \textcolor{red}{Wichtig ist aber, dass in Abschnitt \ref{sect:apt-get}
%    auch die entsprechenden Pakete f�r \software{Python3} installiert
%    wurden!} 
%    \end{bclogo}
%    
%    
%    \cmdPi{\begin{scriptsize}sudo cp /home/pi/RPi-Jukebox-RFID/misc/sampleconfigs/phoniebox-rfid-reader.service.stretch-default.sample\\ /etc/systemd/system/phoniebox-rfid-reader.service\end{scriptsize}}\\
%    \cmdPi{\begin{scriptsize}sudo cp /home/pi/RPi-Jukebox-RFID/misc/sampleconfigs/phoniebox-startup-sound.service.stretch-default.sample\\ /etc/systemd/system/phoniebox-startup-sound.service\end{scriptsize}}\\
%    \cmdPi{\begin{scriptsize}sudo cp /home/pi/RPi-Jukebox-RFID/misc/sampleconfigs/phoniebox-gpio-buttons.service.stretch-default.sample\\ /etc/systemd/system/phoniebox-gpio-buttons.service\end{scriptsize}}\\
%    \cmdPi{\begin{scriptsize}sudo cp /home/pi/RPi-Jukebox-RFID/misc/sampleconfigs/phoniebox-idle-watchdog.service.sample\\ /etc/systemd/system/phoniebox-idle-watchdog.service\end{scriptsize}}
%    
%    Neue Dienste bekannt geben:\\
%    \cmdPi{sudo systemctl daemon-reload}
%    
%    Dienste aktivieren:\\
%    \cmdPi{sudo systemctl enable phoniebox-idle-watchdog}\\
%    \cmdPi{sudo systemctl enable phoniebox-rfid-reader}\\
%    \cmdPi{sudo systemctl enable phoniebox-startup-sound}\\
%    \cmdPi{sudo systemctl enable phoniebox-gpio-buttons}\\
%    \stdout{Created symlink /etc/systemd/system/multi-user.target.wants/\textcolor{red}{<Name>}.service --> /etc/systemd/system/\textcolor{red}{<Name>}.service.}
%    
%    Dienste starten (anstelle eines Reboots):\\
%    \cmdPi{sudo systemctl start phoniebox-idle-watchdog}\\
%    \cmdPi{sudo systemctl start phoniebox-rfid-reader}\\
%    \cmdPi{sudo systemctl start phoniebox-startup-sound}\\
%    \cmdPi{sudo systemctl start phoniebox-gpio-buttons}
%    
%    Status der Dienste anzeigen:\\
%    \cmdPi{sudo systemctl status phoniebox-idle-watchdog}\\
%    \cmdPi{sudo systemctl status phoniebox-rfid-reader}\\
%    \cmdPi{sudo systemctl status phoniebox-startup-sound}\\
%    \cmdPi{sudo systemctl status phoniebox-gpio-buttons}
%    
%    
%    \cmdPi{sudo reboot}
%    
%    TODO!\todo{Fortsetzung folgt\dots}
%    
%    
%    

% chapter3_11.tex -- de (German)
% auto start
%\newpage




%%    \section{\software{systemd:} Autostart der \Bezeichnung-Software einrichten}
%%    \label{sect:systemd}
%%    
%%    \begin{bclogo}[logo = \bclampe, noborder = true]{Hinweis}
%%    Die in den folgenden Kommandos angegebenen Dateien wurden zwar f�r
%%    \os{Raspbian Stretch} vorgesehen, aber auch dort kam schon
%%    \software{systemd} zum Einsatz. Daher ist es unproblematisch, dass hier
%%    die \os{Stretch}-Dateien f�r \os{Raspbian Buster} verwendet werden.
%%    \textcolor{red}{Wichtig ist aber, dass in Abschnitt \ref{sect:apt-get}
%%    auch die entsprechenden Pakete f�r \software{Python3} installiert
%%    wurden!} 
%%    \end{bclogo}
%%    
%%    
%%    \cmdPi{\begin{scriptsize}sudo cp /home/pi/RPi-Jukebox-RFID/misc/sampleconfigs/phoniebox-rfid-reader.service.stretch-default.sample\\ /etc/systemd/system/phoniebox-rfid-reader.service\end{scriptsize}}\\
%%    \cmdPi{\begin{scriptsize}sudo cp /home/pi/RPi-Jukebox-RFID/misc/sampleconfigs/phoniebox-startup-sound.service.stretch-default.sample\\ /etc/systemd/system/phoniebox-startup-sound.service\end{scriptsize}}\\
%%    \cmdPi{\begin{scriptsize}sudo cp /home/pi/RPi-Jukebox-RFID/misc/sampleconfigs/phoniebox-gpio-buttons.service.stretch-default.sample\\ /etc/systemd/system/phoniebox-gpio-buttons.service\end{scriptsize}}\\
%%    \cmdPi{\begin{scriptsize}sudo cp /home/pi/RPi-Jukebox-RFID/misc/sampleconfigs/phoniebox-idle-watchdog.service.sample\\ /etc/systemd/system/phoniebox-idle-watchdog.service\end{scriptsize}}
%%    
%%    Neue Dienste bekannt geben:\\
%%    \cmdPi{sudo systemctl daemon-reload}
%%    
%%    Dienste aktivieren:\\
%%    \cmdPi{sudo systemctl enable phoniebox-idle-watchdog}\\
%%    \cmdPi{sudo systemctl enable phoniebox-rfid-reader}\\
%%    \cmdPi{sudo systemctl enable phoniebox-startup-sound}\\
%%    \cmdPi{sudo systemctl enable phoniebox-gpio-buttons}\\
%%    \stdout{Created symlink /etc/systemd/system/multi-user.target.wants/\textcolor{red}{<Name>}.service --> /etc/systemd/system/\textcolor{red}{<Name>}.service.}
%%    
%%    Dienste starten (anstelle eines Reboots):\\
%%    \cmdPi{sudo systemctl start phoniebox-idle-watchdog}\\
%%    \cmdPi{sudo systemctl start phoniebox-rfid-reader}\\
%%    \cmdPi{sudo systemctl start phoniebox-startup-sound}\\
%%    \cmdPi{sudo systemctl start phoniebox-gpio-buttons}
%%    
%%    Status der Dienste anzeigen:\\
%%    \cmdPi{sudo systemctl status phoniebox-idle-watchdog}\\
%%    \cmdPi{sudo systemctl status phoniebox-rfid-reader}\\
%%    \cmdPi{sudo systemctl status phoniebox-startup-sound}\\
%%    \cmdPi{sudo systemctl status phoniebox-gpio-buttons}
%%    
%%    
%%    %pi@phoniebox1:~/RPi-Jukebox-RFID $ sudo systemctl status phoniebox-idle-watchdog
%%    %. phoniebox-idle-watchdog.service - Phoniebox Idle Watchdog Service
%%    %   Loaded: loaded (/etc/systemd/system/phoniebox-idle-watchdog.service; enabled; vendor preset: enabled)
%%    %   Active: inactive (dead) since Sat 2020-05-02 22:58:40 CEST; 2min 21s ago
%%    %  Process: 2654 ExecStart=/home/pi/RPi-Jukebox-RFID/scripts/idle-watchdog.sh (code=exited, status=0/SUCCESS)
%%    % Main PID: 2654 (code=exited, status=0/SUCCESS)
%%    %
%%    %May 02 22:57:40 phoniebox1 systemd[1]: Started Phoniebox Idle Watchdog Service.
%%    %May 02 22:57:40 phoniebox1 sudo[2656]:       pi : TTY=unknown ; PWD=/home/pi/RPi-Jukebox-RFID ; USER=root ; COMMAND=/usr/bin/atq
%%    %May 02 22:57:40 phoniebox1 sudo[2656]: pam_unix(sudo:session): session opened for user root by (uid=0)
%%    %May 02 22:57:40 phoniebox1 sudo[2656]: pam_unix(sudo:session): session closed for user root
%%    %May 02 22:58:40 phoniebox1 systemd[1]: phoniebox-idle-watchdog.service: Succeeded.
%%    %pi@phoniebox1:~/RPi-Jukebox-RFID $ sudo systemctl status phoniebox-rfid-reader
%%    %. phoniebox-rfid-reader.service - Phoniebox RFID-Reader Service
%%    %   Loaded: loaded (/etc/systemd/system/phoniebox-rfid-reader.service; enabled; vendor preset: enabled)
%%    %   Active: failed (Result: exit-code) since Sat 2020-05-02 22:57:54 CEST; 3min 21s ago
%%    %  Process: 2684 ExecStart=/home/pi/RPi-Jukebox-RFID/scripts/daemon_rfid_reader.py (code=exited, status=1/FAILURE)
%%    % Main PID: 2684 (code=exited, status=1/FAILURE)
%%    %
%%    %May 02 22:57:54 phoniebox1 systemd[1]: phoniebox-rfid-reader.service: Service RestartSec=100ms expired, scheduling restart.
%%    %May 02 22:57:54 phoniebox1 systemd[1]: phoniebox-rfid-reader.service: Scheduled restart job, restart counter is at 5.
%%    %May 02 22:57:54 phoniebox1 systemd[1]: Stopped Phoniebox RFID-Reader Service.
%%    %May 02 22:57:54 phoniebox1 systemd[1]: phoniebox-rfid-reader.service: Start request repeated too quickly.
%%    %May 02 22:57:54 phoniebox1 systemd[1]: phoniebox-rfid-reader.service: Failed with result 'exit-code'.
%%    %May 02 22:57:54 phoniebox1 systemd[1]: Failed to start Phoniebox RFID-Reader Service.
%%    %pi@phoniebox1:~/RPi-Jukebox-RFID $ sudo systemctl status phoniebox-startup-sound
%%    %. phoniebox-startup-sound.service - Phoniebox Startup Sound
%%    %   Loaded: loaded (/etc/systemd/system/phoniebox-startup-sound.service; enabled; vendor preset: enabled)
%%    %   Active: active (exited) since Sat 2020-05-02 22:57:59 CEST; 3min 29s ago
%%    %  Process: 2693 ExecStart=/usr/bin/mpg123 /home/pi/RPi-Jukebox-RFID/shared/startupsound.mp3 (code=exited, status=0/SUCCESS)
%%    % Main PID: 2693 (code=exited, status=0/SUCCESS)
%%    %
%%    %May 02 22:57:59 phoniebox1 systemd[1]: Starting Phoniebox Startup Sound...
%%    %May 02 22:57:59 phoniebox1 mpg123[2693]: High Performance MPEG 1.0/2.0/2.5 Audio Player for Layers 1, 2 and 3
%%    %May 02 22:57:59 phoniebox1 mpg123[2693]:         version 1.25.10; written and copyright by Michael Hipp and others
%%    %May 02 22:57:59 phoniebox1 mpg123[2693]:         free software (LGPL) without any warranty but with best wishes
%%    %May 02 22:57:59 phoniebox1 mpg123[2693]: [src/libmpg123/readers.c:1184] error: Cannot open file /home/pi/RPi-Jukebox-RFID/shared/startupsound.mp3: No such file
%%    %May 02 22:57:59 phoniebox1 mpg123[2693]: main: [src/mpg123.c:708] error: Cannot open /home/pi/RPi-Jukebox-RFID/shared/startupsound.mp3: File access error. (cod
%%    %May 02 22:57:59 phoniebox1 systemd[1]: Started Phoniebox Startup Sound.
%%    %...skipping...
%%    %. phoniebox-startup-sound.service - Phoniebox Startup Sound
%%    %   Loaded: loaded (/etc/systemd/system/phoniebox-startup-sound.service; enabled; vendor preset: enabled)
%%    %   Active: active (exited) since Sat 2020-05-02 22:57:59 CEST; 3min 29s ago
%%    %  Process: 2693 ExecStart=/usr/bin/mpg123 /home/pi/RPi-Jukebox-RFID/shared/startupsound.mp3 (code=exited, status=0/SUCCESS)
%%    % Main PID: 2693 (code=exited, status=0/SUCCESS)
%%    %
%%    %May 02 22:57:59 phoniebox1 systemd[1]: Starting Phoniebox Startup Sound...
%%    %May 02 22:57:59 phoniebox1 mpg123[2693]: High Performance MPEG 1.0/2.0/2.5 Audio Player for Layers 1, 2 and 3
%%    %May 02 22:57:59 phoniebox1 mpg123[2693]:         version 1.25.10; written and copyright by Michael Hipp and others
%%    %May 02 22:57:59 phoniebox1 mpg123[2693]:         free software (LGPL) without any warranty but with best wishes
%%    %May 02 22:57:59 phoniebox1 mpg123[2693]: [src/libmpg123/readers.c:1184] error: Cannot open file /home/pi/RPi-Jukebox-RFID/shared/startupsound.mp3: No such file
%%    %May 02 22:57:59 phoniebox1 mpg123[2693]: main: [src/mpg123.c:708] error: Cannot open /home/pi/RPi-Jukebox-RFID/shared/startupsound.mp3: File access error. (cod
%%    %May 02 22:57:59 phoniebox1 systemd[1]: Started Phoniebox Startup Sound.
%%    %~
%%    %~
%%    %~
%%    %~
%%    %~
%%    %~
%%    %~
%%    %~
%%    %~
%%    %~
%%    %~
%%    %~
%%    %~
%%    %~
%%    %~
%%    %~
%%    %~
%%    %~
%%    %~
%%    %~
%%    %~
%%    %~
%%    %~
%%    %~
%%    %~
%%    %pi@phoniebox1:~/RPi-Jukebox-RFID $ sudo systemctl status phoniebox-gpio-buttons
%%    %. phoniebox-gpio-buttons.service - Phoniebox GPIO Buttons Service
%%    %   Loaded: loaded (/etc/systemd/system/phoniebox-gpio-buttons.service; enabled; vendor preset: enabled)
%%    %   Active: failed (Result: exit-code) since Sat 2020-05-02 22:58:08 CEST; 3min 51s ago
%%    %  Process: 2711 ExecStart=/home/pi/RPi-Jukebox-RFID/scripts/gpio-buttons.py (code=exited, status=203/EXEC)
%%    % Main PID: 2711 (code=exited, status=203/EXEC)
%%    %
%%    %May 02 22:58:08 phoniebox1 systemd[1]: phoniebox-gpio-buttons.service: Service RestartSec=100ms expired, scheduling restart.
%%    %May 02 22:58:08 phoniebox1 systemd[1]: phoniebox-gpio-buttons.service: Scheduled restart job, restart counter is at 5.
%%    %May 02 22:58:08 phoniebox1 systemd[1]: Stopped Phoniebox GPIO Buttons Service.
%%    %May 02 22:58:08 phoniebox1 systemd[1]: phoniebox-gpio-buttons.service: Start request repeated too quickly.
%%    %May 02 22:58:08 phoniebox1 systemd[1]: phoniebox-gpio-buttons.service: Failed with result 'exit-code'.
%%    %May 02 22:58:08 phoniebox1 systemd[1]: Failed to start Phoniebox GPIO Buttons Service.
%%    %pi@phoniebox1:~/RPi-Jukebox-RFID $ 
%%    %
%%    
%%    
%%    
%%    
%%    
%%    
%%    
%%    
%%    
%%    \cmdPi{sudo reboot}
%%    
%%    TODO!\todo{Fortsetzung folgt\dots}
%%    
%%    
%%    

% chapter3_12.tex -- de (German)
% reduction of boot time
\section{Bootzeit reduzieren}

TODO! \todo{Ma�nahmen zur Reduzierung der Bootzeit beschreiben}

Quelle:\\
\url{https://forum-raspberrypi.de/forum/thread/45581-dauer-des-bootvorgaenge-bei-eurer-phoniebox/?postID=414508#post414508}

% chapter3_13.tex -- de (German)
% FAQ
\newpage
\section{Software-FAQ -- oder Hinweise zu "`beliebten"' Fehlern}

In diesem Abschnitt werden nochmals stichpunktartig h�ufig gestellte
Fragen samt ihrer L�sungsm�glichkeiten aufgez�hlt. Hinweise zur
Erweiterung dieses Kapitels werden gerne per e-mail unter
\url{mailto:himself@schlizbaeda.de} entgegengenommen, am liebsten gleich
mit den dazugeh�rigen Antworten \smiley{smile}

\subsection*{Speicherkapazit�t der SD-Karte}
Eigentlich reichen 4GB zur Installation des Betriebssystems \os{Raspbian
Lite}, aber den \software{mpd} darf man dann nicht mehr kompilieren
m�ssen{\dots} (siehe Kapitel \ref{sect:mpd_recompile})\\
\uline{8\,--\,16GB:} Diese Kapazit�t ist ausreichend f�r das
Betriebssystem. Viele Erweiterungen k�nnen \textit{problemlos}
nachinstalliert werden. Zudem ist noch gen�gend Speicher frei f�r die
H�rspiele der Kinder.
\begin{bclogo}[logo = \bclampe, noborder = true]{Hinweis}
Wichtiger als die Speicherkapazit�t ist aber die Zugriffsgeschwindigkeit
auf die Karte. Die sollte mindestens Class 10 betragen. Aber auch hier
kommt es bei der gleichen Angabe zu gro�en messbaren
Geschwindigkeitsunterschieden.\\
Der {\autor} empfiehlt hier tats�chlich, zu den g�ngigen Markenprodukten 
zu greifen.
\end{bclogo}

\subsection*{\os{Raspbian Lite} oder \os{Raspbian Desktop}?}
blabla

\subsection*{Quelldateien von {\autor}s \Bezeichnung}
../files

\subsection*{framps Raspbiback pr�fen}
todo! \todo{testen!}

\subsection*{Startupsound}
Startupsound XOR mpd-Playlist!








%\section{Installation der \Bezeichnung-Software}
%bla bla
%
%
%
%
%
%\section{Software / Installation}
%ReadMe einbinden...
%
%\subsection{offen}
%Reduzierung der Bootzeit\\
%automatisch aus nach 15 min.
%
