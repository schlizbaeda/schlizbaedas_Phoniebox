% chapter3_8.tex -- de (German)
% Using GPIO hardware buttons
\section{Die {\Bezeichnung} mit Tastern an der GPIO-Leiste steuern}
\label{sect:GPIO-buttons}

Urspr�nglich war vorgesehen, bestimmte {\Karte}n mit
Steuerungsfunktionen zu belegen. Dieses Konzept ist aber gerade bei
Kindern ung�nstig, da (zumindest meine) Kinder solche Sachen in den
Tiefen ihrer Zimmer sehr gut verstecken \smiley{smile} und die
{\Bezeichnung} damit zwischenzeitlich(?) unbedienbar wird.\\ 
Daher hat MiczFlor in seinem github-Repository das Kapitel
\url{https://github.com/MiczFlor/RPi-Jukebox-RFID/wiki/Using-GPIO-hardware-buttons}
eingef�gt, in dem die Inbetriebnahme der Steuerung �ber Taster an der
GPIO-Leiste beschrieben wird.

\begin{bclogo}[arrondi = 0.2, logo = \bcinfo, ombre = true, epOmbre = 0.25, couleurOmbre = black!30,blur]{Achtung}
Die Pinbelegung in der Anleitung von MiczFlor ordnet den Tastern Pins
zu, die f�r die digitale Audio�bertragung �ber I2S zum HifiBerry
{\miniamp} ben�tigt werden! Daher muss das Pythonskript f�r die
Abfrage der Taster stark angepasst werden!
\end{bclogo}

F�r den Zugriff auf die GPIOs wird die Python-Bibliothek
\software{gpiozero} verwendet, von der es jeweils eine eigene Variante
f�r Python2 und Python 3 gibt. Installation der Bibliothek �ber
folgendes Kommando:\\
\cmdPi{sudo apt-get install python3-gpiozero python-gpiozero}
\comment{f�r Python2 \textbf{und} Python3!}

Zur Abfrage der Taster wird das Pythonskript
\filenam{/home/pi/RPi-Jukebox-RFID/scripts/gpio-buttons.py} verwendet.
Die Originalvorlage k�nnte mit folgendem Kommando installiert werden:\\
\cmdPi{\begin{scriptsize}sudo cp /home/pi/RPi-Jukebox-RFID/misc/sampleconfigs/gpio-buttons.py.sample\\ /home/pi/RPi-Jukebox-RFID/scripts/gpio-buttons.py\end{scriptsize}}
\comment{funktioniert nicht mit dem \miniamp!}\\
Allerdings kollidiert die Pinvergabe mit der Pinzuordnung des HifiBerry
{\miniamp}s!

Bei Einbau des {\miniamp}s muss die von {\autor} angepasste Datei
verwendet werden! Diese Datei entspricht der Pinbelegung der
Lochrasterplatine aus Kapitel \ref{sect:prototypingboard}:\\
\cmdPC{scp ./files/GPIO/gpio-buttons.py pi@phoniebox1:/home/pi/RPi-Jukebox-RFID/scripts}

\cmdPi{nano /home/pi/RPi-Jukebox-RFID/scripts/gpio-buttons.py}\\
%\begin{verbatim}
\verb|#shut = Button(3,hold_time=2)  # --> Pin 5 # no longer necessary due to OnOffShim|\\
\verb|vol0 = Button(13,pull_up=True) # --> Pin 33|\\
\verb|volU = Button(12,pull_up=True,hold_time=0.3,hold_repeat=True) # --> Pin 32|\\
\verb|volD = Button(6,pull_up=True,hold_time=0.3,hold_repeat=True)  # --> Pin 31|\\
\verb|next = Button(7,pull_up=True)  # --> Pin 26|\\
\verb|prev = Button(8,pull_up=True)  # --> Pin 24|\\
\verb|halt = Button(5,pull_up=True)  # --> Pin 29|\\
\verb|#reco = Button(6, pull_up=True) # Choose GPIO to fit your hardware|\\
\verb|#play = Button(12,pull_up=True) # Choose GPIO to fit your hardware|
%\end{verbatim}

siehe auch Tabelle \ref{tab:gpio_rpi}
%\comment{{\autor}s Variante}




%bla
%
%
%\cmdPi{\begin{scriptsize}sudo cp /home/pi/RPi-Jukebox-RFID/misc/sampleconfigs/phoniebox-idle-watchdog.service.sample\\ /etc/systemd/system/phoniebox-idle-watchdog.service\end{scriptsize}}
%
%bla






%\begin{bclogo}[logo = \bclampe, noborder = true]{Hinweis}
%Die in den folgenden Kommandos angegebenen Dateien wurden zwar f�r
%\os{Raspbian Stretch} vorgesehen, aber auch dort kam schon
%\software{systemd} zum Einsatz. Daher ist es unproblematisch, dass hier
%die \os{Stretch}-Dateien f�r \os{Raspbian Buster} verwendet werden.
%\textcolor{red}{Wichtig ist aber, dass in Abschnitt \ref{sect:apt-get}
%auch die entsprechenden Pakete f�r \software{Python3} installiert
%wurden!} 
%\end{bclogo}
%
%
%\cmdPi{\begin{scriptsize}sudo cp /home/pi/RPi-Jukebox-RFID/misc/sampleconfigs/phoniebox-rfid-reader.service.stretch-default.sample\\ /etc/systemd/system/phoniebox-rfid-reader.service\end{scriptsize}}\\
%\cmdPi{\begin{scriptsize}sudo cp /home/pi/RPi-Jukebox-RFID/misc/sampleconfigs/phoniebox-startup-sound.service.stretch-default.sample\\ /etc/systemd/system/phoniebox-startup-sound.service\end{scriptsize}}\\
%\cmdPi{\begin{scriptsize}sudo cp /home/pi/RPi-Jukebox-RFID/misc/sampleconfigs/phoniebox-gpio-buttons.service.stretch-default.sample\\ /etc/systemd/system/phoniebox-gpio-buttons.service\end{scriptsize}}\\
%\cmdPi{\begin{scriptsize}sudo cp /home/pi/RPi-Jukebox-RFID/misc/sampleconfigs/gpio-buttons.py.sample\\ /home/pi/RPi-Jukebox-RFID/scripts/gpio-buttons.py\begin{scriptsize}}
%
%
%\cmdPi{\begin{scriptsize}sudo cp /home/pi/RPi-Jukebox-RFID/misc/sampleconfigs/phoniebox-idle-watchdog.service.sample\\ /etc/systemd/system/phoniebox-idle-watchdog.service\end{scriptsize}}
%
%Neue Dienste bekannt geben:\\
%\cmdPi{sudo systemctl daemon-reload}
%
%Dienste aktivieren:\\
%\cmdPi{sudo systemctl enable phoniebox-idle-watchdog}\\
%\cmdPi{sudo systemctl enable phoniebox-rfid-reader}\\
%\cmdPi{sudo systemctl enable phoniebox-startup-sound}\\
%\cmdPi{sudo systemctl enable phoniebox-gpio-buttons}\\
%\stdout{Created symlink /etc/systemd/system/multi-user.target.wants/\textcolor{red}{<Name>}.service --> /etc/systemd/system/\textcolor{red}{<Name>}.service.}
%
%Dienste starten (anstelle eines Reboots):\\
%\cmdPi{sudo systemctl start phoniebox-idle-watchdog}\\
%\cmdPi{sudo systemctl start phoniebox-rfid-reader}\\
%\cmdPi{sudo systemctl start phoniebox-startup-sound}\\
%\cmdPi{sudo systemctl start phoniebox-gpio-buttons}
%
%Status der Dienste anzeigen:\\
%\cmdPi{sudo systemctl status phoniebox-idle-watchdog}\\
%\cmdPi{sudo systemctl status phoniebox-rfid-reader}\\
%\cmdPi{sudo systemctl status phoniebox-startup-sound}\\
%\cmdPi{sudo systemctl status phoniebox-gpio-buttons}
%
%
%\cmdPi{sudo reboot}


