% chapter1.tex -- en (English)
Sorry, but the English version isn't finished yet!
%\chapter{Preface}
%
%Thank you for interesting in {\autor}'s {\Bezeichnung}.\\
%This is a control unit for PCs, which can emulate keyboard and mouse
%actions and is inspired on the gaming controllers of various game 
%consoles. It is based on the Arduino board \textit{Esplora} which is
%technically a \textit{Arduino Leonardo} and it already contains some
%sensors and actuators. The printed circuit board was designed roughly in
%the form of a gaming controller.
%
%\url{https://store.arduino.cc/arduino-esplora}
%
%\section{Legal Notes}
%When designing the {\Bezeichnung}, it was taken care to use only 
%software that is provided under a free license such as GNU GPL or 
%similar or that has been released into the public domain.
%
%\subsection*{Trademarks}
%Some names used in this document may be trademarks. The use of these 
%trademarks by third parties for their purposes may infringe the rights 
%of the holders.
%
%\subsection*{Links}
%This manual contains links to external sites on the internet. Despite
%linking the author {\autor} does not appropriate these contents, since
%they are not within his sphere of influence! At the time of linking
%there were no illegal contents noticeable. It is not reasonable for the
%author to check the links permanently for any changes that might violate
%the law. However, if current or future content should be illegal, the 
%author may be contacted by e-mail to
%\url{mailto:himself@schlizbaeda.de}.
%Appropriate actions will then be taken to remove the affected link(s).
%
%\subsection*{Licensing of this project's components}
%\uline{Code:}\\
%Most Arduino code examples are in the public domain and may be used
%by any person. The code for {\Bezeichnung} was developed from the
%templates\\
%\url{https://www.arduino.cc/en/Tutorial/EsploraJoystickMouse},\\
%\url{https://www.arduino.cc/en/Tutorial/EsploraKart} and\\
%\url{https://github.com/circuit69/EsploraTinkerkit}\\
%Due to many adjustments and improvements the author decided to publish
%his work under the GNU GPL v3:\\
%\url{https://www.gnu.org/licenses/gpl-3.0.en.html}\\
%\includegraphics[height=35px]{GPLv3.png}
%
%\uline{Housing:}\\
%A CAD file collection for 3D printing was found at
%\url{https://www.thingiverse.com/thing:45880}. This data is published
%under CC BY-NC 3.0
%(\url{http://creativecommons.org/licenses/by-nc/3.0/}). 
%This license does not allow commercial use.\\
%\includegraphics[height=35px]{CC-BY-NC.png}
%
%
%\subsection*{Image Copyright}
%All pictures, photographs and technical drawings were created by the
%author himself. They are published under the \textit{Creative Commons}
%license \textbf{CC BY-SA 3.0}. Therefore anyone may use these images
%in original form or adapted by giving name attribution of the author.\\
%\includegraphics[height=35px]{CC-BY-SA.png}
%
%The Arduino logo is in the public domain:\\
%\includegraphics[height=35px]{arduino.png}
%
%
%\newpage
%\section{Remarks to a genderwise correctly written text}
%Grammatical gender differentiation is very unusual in English language
%compared to German language. In  German speaking countries there is
%currently a trend of \textit{political correctness} to mix the
%grammatical gender with the biological sex. This results in
%\textit{genderwise correctly} written texts which are sometimes really
%difficult to read.\\
%This translation isn't affected by this German gender trend anyway. So
%this may be regarded as a typical German problem \smiley{wink}
%
%\section{Achnowledgements}
%{\autor} would like to welcome the user \textbf{@dale}
%(\url{https://forum-raspberrypi.de/user/23726-dale/}) from the German
%Raspberry Pi forum (\url{https://forum-raspberrypi.de}):\\
%He printed the housing on his 3D printer.
%
%\section{Abstract} % Brief Description
%The {\Bezeichnung} is a control unit with USB2.0 connection inspired by
%the game console based gaming controllers. Instead of proprietary
%signals this device transmits keyboard and mouse commands when the user
%actuates the switches of the device.
%
%Due to its open-source and free programming, it can be adapted to many
%applications in the source code by assigning the desired keyboard codes
%or mouse commands to the control elements (sensors) of the device.
%
%There are eight different assignments (so-called modes) for several
%applications which can be switched during operation. The selected 
%assignment is indicated by the colour of the three-colour LED on the
%Esplora board.
%
%The software of the {\Bezeichnung} emulates a USB keyboard and a USB
%mouse and can be used immediately on most of the PC operating systems
%like \textit{Windows} or \textit{GNU/Linux} as well as on a {\RPi} under
%\textit{Raspbian} without any additional driver installation.
%
